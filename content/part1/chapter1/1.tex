下载和构建LLVM代码非常简单,不需要付费工具。需要准备:

\begin{itemize}
\item
基于Unix的操作系统(Linux, Darwin)

\item
命令行git

\item
构建工具:CMake和Ninja
\end{itemize}

我们将使用调试器作为源代码调查工具。LLVM有自己的调试器,LLDB。我们将从LLVM仓库中构建它,作为我们的第一个工具:\url{https://github.com/llvm/llvm-project.git}。

构建过程都包含两个步骤。第一个是项目配置,最后一个是构建本身。LLVM使用CMake作为项目配置工具,还可以使用其他构建工具,如Unix Makefiles和Ninja。可以为流行的IDE,如Visual Studio和XCode生成项目文件。这里使用Ninja作为构建工具,它可以加快构建过程,并且大多数LLVM开发者都使用它。可以在以下链接找到有关工具的更多信息:\url{https://llvm.org/docs/GettingStarted.html}。

本章的源代码位于本书GitHub仓库的\textbf{chapter1}文件夹中:\url{https://github.com/PacktPublishing/Clang-Compiler-Frontend-Packt/tree/main/chapter1}。

\mySubsubsection{1.1.1.}{项目配置工具CMake}

CMake是一个开源的、跨平台的构建系统生成器。自2013年发布的版本3.3起,一直作为LLVM的主要构建工具。

LLVM使用CMake之前使用的时autoconf,这是一个生成配置脚本的工具,可以在类Unix系统中用来构建和安装软件。然而,autoconf有几个限制,比如使用和维护困难,以及对跨平台构建的支持不佳。CMake选为autoconf的替代品,解决了这些限制,并且更易于使用和维护。

除了用作LLVM的构建系统外,CMake还在许多其他软件项目中使用,包括Qt、OpenCV和Google Test。

\mySubsubsection{1.1.2.}{Ninja作为构建工具}

Ninja是一个小型构建系统,专注于速度。可与构建生成器一起使用,如CMake,后者生成描述项目构建规则的构建文件。

Ninja的主要优势之一是其速度,能够比其他构建系统,如Unix Makefiles,执行构建得更快,只重新构建完成构建所需的最小文件集。因为它跟踪构建目标之间的依赖关系,并且只重新构建过时的目标。

此外,Ninja简单易用。有一个小而直接的命令行界面,使用的构建文件是简单易读的文本文件。

当构建系统的速度是一个关注点,并且需要一个简单易用的工具时,Ninja是一个不错的选择。

Ninja最有用的选项是-j,这个选项允许指定并行运行的线程数。

我们的下一个目标是下载LLVM代码并调查项目结构,还需要设置构建过程所需的必要工具,并为未来使用LLVM代码的实验建立环境。这将确保我们拥有进行高效工作的工具和依赖项。






























