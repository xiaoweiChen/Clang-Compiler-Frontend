让我们从了解一些关于LLVM的基础信息开始,包括项目历史及其结构。

\mySubsubsection{1.2.1.}{LLVM简史}

Clang编译器是LLVM项目的一部分。该项目始于2000年,由Chris Lattner和Vikram Adve在伊利诺伊大学厄巴纳-香槟分校作为他们的项目而启动\footnote{Chris Lattner and Vikram Adve. LLVM: A Compilation Framework for Lifelong Program Analysis \& Transformation. 2004年国际代码生成与优化研讨会(CGO'04)论文集, 2004年3月。}。

LLVM最初被设计为一个下一代代码生成基础设施,可以用来构建多种编程语言的优化编译器。然而,它已经发展成为一个全功能的平台,可以用来构建各种工具,包括调试器、分析器和静态分析工具。

LLVM在软件行业得到了广泛采用,许多公司和组织使用它来构建各种工具和应用。它也用于学术研究和教学,并启发了其他领域类似项目的发展。

当苹果公司在2005年雇佣了Chris Lattner并组建了一个团队来开发LLVM时,该项目获得了额外的推动。LLVM成为了苹果公司创建的开发工具(XCode)的重要组成部分。

最初,GNU编译器集合(GCC)用作LLVM的C/C++前端。但这存在一些问题。其中一个问题是与GNU通用公共许可证(GPL)相关,它阻止了前端在某些专有项目中的使用。另一个缺点是当时GCC对Objective-C的支持有限,这对苹果来说很重要。Chris Lattner在2006年启动了Clang项目来解决这些问题。

Clang最初被设计为一个统一的解析器,用于C语言家族,包括C、Objective-C、C++和Objective-C++。这种统一旨在通过为多种语言使用单一的前端实现来简化维护,而不是为每种语言维护多个实现。该项目很快获得了成功。Clang和LLVM成功的主要原因之一是它们的模块化。LLVM中的所有内容都是一个库,包括Clang。这为基于Clang和LLVM创建许多惊人的工具打开了机会,例如稍后将在本书中介绍的clang-tidy和clangd)。

LLVM和Clang具有非常清晰的架构,并且是用C++编写的。这使得任何C++开发者都可以研究和使用它。我们可以看到围绕LLVM建立起来的庞大社区以及其使用的飞速增长。

\mySubsubsection{1.2.2.}{操作系统支持}

我们在这里计划专注于个人电脑的操作系统,如Linux、Darwin和Windows。另一方面,Clang不仅限于个人电脑,还可以用来为移动平台,如iOS和不同的嵌入式系统编译代码。

\mySamllsection{Linux}

GCC是Linux上的默认开发工具集,特别是gcc(用于C程序)和g++(用于C++程序)是默认的编译器。Clang也可以用来在Linux上编译源代码。此外,它模仿gcc并支持其大多数选项。然而,对于一些GNU工具,LLVM的支持可能是有限的;例如,GNU Emacs不支持将LLDB作为调试器。但尽管如此,Linux是最适合LLVM开发和调查的操作系统,因此我们将主要使用这个操作系统(Fedora 39)来进行后续示例。

\mySamllsection{Darwin (macOS)}

Clang被认为是Darwin的主要构建工具。整个构建基础设施基于LLVM,Clang是默认的C/C++编译器。开发者工具,如调试器(LLDB),也来自LLVM。您可以从XCode获取主要的开发者工具,这些工具基于LLVM。然而,您可能需要安装额外的命令行工具,如CMake和Ninja,可以作为单独的包安装,或通过包管理系统如MacPorts或Homebrew安装。

例如,可以使用Homebrew以下列方式获取CMake:

\begin{shell}
$ brew install cmake
\end{shell}

或者MacPorts:

\begin{shell}
$ sudo port install cmake
\end{shell}

\mySamllsection{Windows}

在Windows上,Clang可以用作命令行编译器,也可以作为更大开发环境的一部分,如Visual Studio。Windows上的Clang包括对Microsoft Visual C++ (MSVC) ABI的支持,因此您可以使用Clang来编译使用Microsoft C运行时库(CRT)和C++标准模板库(STL)的程序。Clang也支持GCC的许多相同语言特性,因此在许多情况下,它可以作为GCC在Windows上的直接替代品。

值得一提的是clang-cl \footnote{LLVM社区. MSVC兼容性. 2023. URL \url{https://clang.llvm.org/docs/MSVCCompatibility.html}.}。它是一个为Clang设计的命令行编译器驱动程序,旨在用作MSVC编译器cl.exe的直接替代品。它是作为Clang编译器的一部分引入的,并且是为了与LLVM工具链一起使用而创建的。

与cl.exe一样,clang-cl用于Windows程序的构建过程,并且它支持MSVC编译器的许多相同命令行选项。它可以在Windows上用于编译C、C++和Objective-C代码,并且还可以用于链接对象文件和库以创建可执行程序或动态链接库(DLL)。

Windows的开发过程与类Unix系统不同,它需要额外的特定细节,这可能会使本书的材料变得相当复杂。为了避免这种复杂性,我们的主要目标是专注于基于Unix的系统,如Linux和Darwin,并且我们将在本书中省略Windows特定的示例。


\mySubsubsection{1.2.3.}{LLVM/Clang项目结构}


Clang源代码是LLVM单体仓库(monorepo)的一部分。LLVM从2019年开始使用单体仓库作为其过渡到Git的一部分\footnote{LLVM社区. 将LLVM项目迁移到GitHub. 2019. URL \url{https://llvm.org/docs/Proposals/GitHubMove.html}.}。这个决定是由几个因素驱动的,比如更好的代码重用、提高效率和协作。因此,您可以在一个地方找到所有的LLVM项目。如前言所述,本书我们将使用LLVM版本18.x。用以下命令下载:

\begin{shell}
$ git clone https://github.com/llvm/llvm-project.git -b release/18.x
$ cd llvm-project
\end{shell}

\begin{center}
图1.1: 获取LLVM代码
\end{center}


\begin{myNotic}{重要提示}
版本18是LLVM的最新版本,预计将于2024年3月发布。本书基于2024年1月23日的版本,当时创建了发布分支。
\end{myNotic}


本书将使用的llvm-project最重要的部分如图1.2所示。

\myGraphic{0.6}{content/part1/chapter1/images/1.png}{图1.2: LLVM项目树}

其中包括

\begin{itemize}
\item
lld : LLVM链接器工具。您可能希望将其用作标准链接器工具的替代品,例如GNU ld

\item
llvm : LLVM项目的通用库

\item
clang :  Clang驱动程序和前端

\item
clang-tools-extra :本书第二部分将介绍的不同Clang工具
\end{itemize}

大多数项目具有图1.3所示的结构。

\myGraphic{0.5}{content/part1/chapter1/images/2.png}{图1.3: 典型的LLVM项目结构}

LLVM项目,如clang或llvm,通常包含两个主要文件夹:include和lib。include文件夹包含项目接口(头文件),而lib文件夹包含实现。每个LLVM项目都有各种不同的测试,可以分为两个主要组:位于unittests文件夹中的单元测试,使用Google Test框架实现,以及使用LLVM集成测试器(LIT)框架实现的端到端测试。

对我们来说最重要的项目是clang和clang-tools-extra。clang文件夹包含前端和驱动程序。

\begin{myNotic}{重要提示}
编译器驱动程序用于运行编译的不同阶段(解析、优化、链接等)。
\end{myNotic}


例如,词法分析器的实现位于clang/lib/Lex文件夹中。您还可以看到clang/test文件夹,其中包含端到端测试,以及clang/unittest文件夹,其中包含前端和驱动程序的单元测试。

另一个重要的文件夹是clang-tools-extra。它包含一些基于不同Clang库的工具。它们如下:


\begin{itemize}
\item
clang-tools-extra/clangd : 为VSCode等IDE提供导航信息的语言服务器

\item
clang-tools-extra/clang-tidy : 一个强大的代码检查框架,包含数百种不同的检查

\item
clang-tools-extra/clang-format :  一个代码格式化工具
\end{itemize}

获取源代码并设置构建工具后,我们就可以准备编译LLVM源代码了。























