为了第一个测试项目,创建一个简单的Clang工具,该工具运行编译器,并检查提供的源文件的语法。先创建一个所谓的树外LLVM项目,即一个将使用LLVM但位于主LLVM源树外的项目。

创建项目需要执行以下几项操作:

\begin{itemize}
\item
必须构建和安装所需的LLVM库和头文件

\item
需要为测试项目创建一个构建配置文件

\item
使用LLVM的源代码必须创建
\end{itemize}

从第一步开始,安装Clang支持库和头文件。使用以下CMake配置命令:

\begin{shell}
cmake -G Ninja -DCMAKE_BUILD_TYPE=Debug -DCMAKE_INSTALL_PREFIX=../install -DLLVM_TARGETS_TO_BUILD="X86" -DLLVM_ENABLE_PROJECTS="clang" -DLLVM_USE_LINKER=gold -DLLVM_USE_SPLIT_DWARF=ON -DBUILD_SHARED_LIBS=ON ../llvm
\end{shell}

\begin{center}
图 1.12: LLVM CMake配置,用于简单的语法检查Clang工具
\end{center}

这里只启用了一个项目:clang。所有其他选项都是我们调试构建的标准配置。该命令必须从LLVM源树中创建的构建文件夹中运行。

\begin{myNotic}{重要提示}
图1.12中指定的配置将在整本书中作为默认构建配置使用。
\end{myNotic}

使用动态库的配置,除了减少大小外,还有简化依赖项指定的优点。只需要指定项目直接依赖的动态库,动态链接器将处理其余部分。

所需的库和头文件可以通过以下命令安装:

\begin{shell}
$ ninja install
\end{shell}

库和头文件将安装到通过CMAKE\_INSTALL\_PREFIX选项指定的文件夹中。

再为项目创建两个文件:

\begin{itemize}
\item
CMakeLists.txt: 项目配置文件

\item
TestProject.cpp: 项目源代码
\end{itemize}

项目配置文件,CMakeLists.txt,将通过LLVM\_HOME环境变量接受LLVM安装文件夹的路径。该文件如下所示:

\begin{cmake}
cmake_minimum_required(VERSION 3.16)
project("syntax-check")

if ( NOT DEFINED ENV{LLVM_HOME})
  message(FATAL_ERROR "$LLVM_HOME is not defined")
else()
  message(STATUS "$LLVM_HOME found: $ENV{LLVM_HOME}")
  set(LLVM_HOME $ENV{LLVM_HOME} CACHE PATH "Root of LLVM installation")
  set(LLVM_LIB ${LLVM_HOME}/lib)
  set(LLVM_DIR ${LLVM_LIB}/cmake/llvm)
  find_package(LLVM REQUIRED CONFIG)
  include_directories(${LLVM_INCLUDE_DIRS})
  link_directories(${LLVM_LIBRARY_DIRS})
  set(SOURCE_FILES SyntaxCheck.cpp)
  add_executable(syntax-check ${SOURCE_FILES})
  set_target_properties(syntax-check PROPERTIES COMPILE_FLAGS "-fno-rtti")
  target_link_libraries(syntax-check
    LLVMSupport
    clangBasic
    clangFrontend
    clangSerialization
    clangTooling
  )
endif()
\end{cmake}

\begin{center}
图 1.13: 用于简单语法检查Clang工具的CMake文件
\end{center}

\begin{itemize}
\item
第2行:指定了项目名称(syntax-check)。这也是可执行文件的名称。

\item
第4-7行:测试LLVM\_HOME环境变量。

\item
第10行:设置了LLVM CMake帮助器的路径。

\item
第11行:从第10行指定的路径加载LLVM CMake包。

\item
第14行:指定了编译的源文件。

\item
第16行:设置了相应的编译标志:-fno-rtti。当LLVM没有构建RTTI时,为了减少代码和可执行文件大小 \footnote{LLVM社区. LLVM编码标准. 2023. URL \url{https://llvm.org/docs/CodingStandards.html}},这个标志是必需的。

\item
第18-22行:指定要链接到我们程序的必需库
\end{itemize}

工具源码如下:

\begin{cpp}
#include "clang/Frontend/FrontendActions.h" // clang::SyntaxOnlyAction
#include "clang/Tooling/CommonOptionsParser.h"
#include "clang/Tooling/Tooling.h"
#include "llvm/Support/CommandLine.h" // llvm::cl::extrahelp

namespace {
  llvm::cl::OptionCategory TestCategory("Test project");
  llvm::cl::extrahelp
    CommonHelp(clang::tooling::CommonOptionsParser::HelpMessage);
} // namespace

int main(int argc, const char **argv) {
  llvm::Expected<clang::tooling::CommonOptionsParser> OptionsParser =
    clang::tooling::CommonOptionsParser::create(argc, argv, TestCategory);
  if (!OptionsParser) {
    llvm::errs() << OptionsParser.takeError();
    return 1;
  }
  clang::tooling::ClangTool Tool(OptionsParser->getCompilations(),
                                 OptionsParser->getSourcePathList());
  return Tool.run(
    clang::tooling::newFrontendActionFactory<clang::SyntaxOnlyAction>()
            .get());
}
\end{cpp}

\begin{center}
图 1.14: SyntaxCheck.cpp
\end{center}

\begin{itemize}
\item
第7-9行:大多数编译器工具都有一套相同的命令行参数。LLVM命令行库 \footnote{LLVM社区. CommandLine 2.0库手册. 2023. URL \url{https://llvm.org/docs/CommandLine.html}} 提供了一些API来处理编译器命令选项。第7行设置了该库,在第8-10行设置了相应的帮助信息。

\item
第13-18行:解析命令行参数。

\item
第19-24行:创建并运行Clang工具。

\item
第22-23行:使用clang::SyntaxOnlyAction前端动作,将在输入文件上运行语法和语义检查。
\end{itemize}

必须指定LLVM安装文件夹的路径来构建工具。如之前所述,路径必须通过LLVM\_HOME环境变量指定。配置命令(见图1.12)将路径指定为LLVM项目源树中的安装文件夹,所以可以按照以下方式构建工具:

\begin{shell}
export LLVM_HOME=<...>/llvm-project/install
mkdir build
cd build
cmake -G Ninja ..
ninja
\end{shell}

\begin{center}
图 1.15: syntax-check构建命令
\end{center}

可以按照以下方式运行工具:

\begin{shell}
$ cd build
$ ./syntax-check --help
USAGE: syntax-check [options] <source0> [... <sourceN>]
...
\end{shell}

\begin{center}
图 1.16: syntax-check –help 的输出
\end{center}

如果在一个C++源文件上运行程序,它将直接退出;但在一个有语法错误的C++文件上运行,会产生错误消息:

\begin{shell}
$ ./syntax-check mainbroken.cpp -- -std=c++17
mainbroken.cpp:2:11: error: expected ';' after return statement
  return 0
          ^
          ;
1  error generated.
Error while processing mainbroken.cpp.
\end{shell}

\begin{center}
图 1.17: 在语法错误文件上运行syntax-check
\end{center}

图1.17中使用'-{}-'将附加参数传递给编译器,具体指定C++17,选项'-std=c++17'。

还可以使用LLDB调试器运行工具:

\begin{shell}
$  <...>/llvm-project/install/bin/lldb \
                  ./syntax-check \
                  --           \
                  main.cpp     \
                  -- -std=c++17
\end{shell}

\begin{center}
图 1.18: 调试器下运行syntax-check
\end{center}

以syntax-check为主程序,将main.cpp源文件作为工具的参数(图1.18)。还可以向syntax-check可执行文件传递编译标志(-std=c++17)。

可以设置断点,并按照以下方式运行程序:

\begin{shell}
1  (lldb) b clang::ParseAST
2  ...
3  (lldb) r
4  ...
5  Running without flags.
6  Process 608249 stopped
7  * thread #1, name = 'syntax-check', stop reason = breakpoint 1.1
8    frame #0: ... clang::ParseAST(...) at ParseAST.cpp:117:3
9     114
10    115  void clang::ParseAST(Sema &S, bool PrintStats, bool SkipFunctionBodies) {
11    116    // Collect global stats on Decls/Stmts (until we have a module streamer).
12 -> 117    if (PrintStats) {
13    118     Decl::EnableStatistics();
14    119     Stmt::EnableStatistics();
15    120    }
16 (lldb) c
17 Process 608249 resuming
18 Process 608249 exited with status = 0 (0x00000000)
19 (lldb)
\end{shell}

\begin{center}
图 1.19: LLDB的交互演示,用于Clang工具测试项目
\end{center}

在clang::ParseAST函数中设置了一个断点(图1.19,第1行)。该函数是源代码解析的主要入口点。第3行运行程序,并在第16行继续执行断点后的程序。

本书的后续章节中,将使用相同的调试技术来研究Clang的源码。






