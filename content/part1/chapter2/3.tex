讨论编译器时,通常指的是一个命令行工具,启动和管理编译过程。例如,要使用GNU编译器集合,必须调用gcc来启动编译过程。同样,要使用Clang编译C++程序,也必须将clang作为编译器来调用。控制编译过程的程序称为驱动程序,驱动程序协调整个编译过程的不同阶段,并将它们连接在一起。本书中,我们将重点关注LLVM,并使用Clang作为编译过程的驱动程序。

对于读者来说,可能会有点混乱。因为"Clang"这个词既用来指代编译器前端,也用来指代编译驱动程序。相比之下,在其他编译器中,驱动程序和C++编译器可以是分开的可执行文件,而"Clang"是一个可执行文件,既作为驱动程序,也作为编译器前端。若要仅将Clang用作编译器前端,必须向其传递特殊选项-cc1。

\mySubsubsection{2.3.1.}{示例}

我们将使用简单的"Hello world!"示例程序来与Clang驱动程序进行实验。主源文件名为hello.cpp。该文件实现了一个简单的C++程序,将打印"Hello world!":

\begin{cpp}
#include <iostream>

int main() {
  std::cout << "Hello world!" << std::endl;
  return 0;
}
\end{cpp}

\begin{center}
图2.13:示例程序:hello.cpp
\end{center}

可以使用以下命令编译源代码:

\begin{shell}
$ <...>/llvm-project/install/bin/clang hello.cpp -o /tmp/hello -lstdc++
\end{shell}

\begin{center}
图2.14:hello.cpp的编译
\end{center}

使用clang可执行文件作为编译器,并指定了-lstdc++库选项,因为使用了标准C++库中的头文件,还使用-o选项指定了可执行文件的输出位置(/tmp/hello)。

\mySubsubsection{2.3.2.}{编译阶段}

示例程序使用了两个输入。第一个是我们的源码,第二个是标准C++库的动态库。Clang驱动程序应该将输入组合在一起,通过编译过程的不同阶段传递,并最终在目标平台上提供可执行文件。

Clang使用与图2.2所示相同的典型编译器工作流程,可以使用-ccc-print-phases附加参数要求Clang显示这些阶段:

\begin{shell}
$ <...>/llvm-project/install/bin/clang hello.cpp -o /tmp/hello -lstdc++ \
  -ccc-print-phases
\end{shell}

\begin{center}
图2.15:用于打印hello.cpp编译阶段的命令
\end{center}

该命令的输出如下:

\begin{shell}
            +- 0: input, "hello.cpp", c++
         +- 1: preprocessor, {0}, c++-cpp-output
      +- 2: compiler, {1}, ir
   +- 3: backend, {2}, assembler
+- 4: assembler, {3}, object
|- 5: input, "1%dM", object
6 : linker, {4, 5}, im
\end{shell}

\begin{center}
图2.16:hello.cpp的编译阶段
\end{center}

可以如图2.17所示可视化输出。

如图 2.17 中所见,驱动程序接收一个输入文件 hello.cpp,这是一个 C++ 文件。该文件经过预处理器处理后,获得了预处理器输出(标记为 c++-cpp-output)。结果由编译器编译成中间表示 (IR) 形式,然后后端将其转换为汇编形式。这种形式随后被转化为一个目标文件,最终的目标文件与另一个对象(libstdc++)合并,产生最终的可执行文件(image)。

\newpage

\myGraphic{0.8}{content/part1/chapter2/images/17.png}{图2.17:Clang驱动程序阶段}

\mySubsubsection{2.3.3.}{使用工具}

这些阶段组合成了几次工具执行,Clang 驱动程序调用不同的程序来生成最终的可执行文件。例子中,调用了 Clang 编译器和 ld 链接器。这两个程序都需要其他的参数,这些参数由驱动程序设置。

例如,示例程序 hello.cpp 包含了以下头文件:

\begin{cpp}
#include <iostream>
...
\end{cpp}

\begin{center}
图 2.18: iostream 头文件在 hello.cpp 中的位置
\end{center}

调用编译时,并未指定参数(比如搜索路径,例如 -I)。但不同的架构和操作系统,可能有不同的头文件搜索位置。

在 Fedora 39 上,该头文件位于 /usr/include/c++/13/iostream 文件夹中。我们可以使用 -\#\#\# 选项来查看驱动程序执行的过程以及使用的参数:

\begin{shell}
$ <...>/llvm-project/install/bin/clang hello.cpp -o /tmp/hello -lstdc++ -###
\end{shell}

\begin{center}
图 2.19: 打印 hello.cpp 编译工具执行命令
\end{center}

此命令的输出非常详尽,这里省略了一部分内容,如图 2.20所示。

\begin{shell}
1   clang version 18.1.0rc (https://github.com/llvm/llvm-project.git ...)
2   "<...>/llvm-project/install/bin/clang-18"
3     "-cc1" ... \
4     "-internal-isystem" \
5     "/usr/include/c++/13" ... \
6     "-internal-isystem" \
7     "/usr/include/c++/13/x86_64-redhat-linux" ... \
8     "-internal-isystem" ... \
9     "<...>/llvm-project/install/lib/clang/18/include" ... \
10    "-internal-externc-isystem" \
11    "/usr/include" ... \
12    "-o" "/tmp/hello-XXX.o" "-x" "c++" "hello.cpp"
13  ".../bin/ld" ... \
14    "-o" "/tmp/hello" ... \
15    "/tmp/hello-XXX.o" \
16    "-lstdc++" ...
\end{shell}

\begin{center}
图 2.20: Clang 驱动程序工具执行,系统为 Fedora 39。
\end{center}

如图 2.20 所示,驱动程序启动了两个进程:带有 -cc1 标志的 clang-18(参见第 2 至 12 行)和链接器 ld(参见第 13 至 16 行)。Clang 编译器隐式地接收了多个搜索路径,如第 5、7、9 和 11 行所示。这些路径对于包含测试程序中的 iostream 头文件是必要的。

第一个可执行文件的输出(/tmp/hello-XXX.o)作为第二个可执行文件的输入(参见第 12 和 15 行)。链接器的参数为 -lstdc++ 和 -o /tmp/hello ,而第一个参数(hello.cpp)是为编译器调用(第一个可执行文件)提供。

\myGraphic{0.4}{content/part1/chapter2/images/21.png}{图 2.21: Clang 驱动程序工具执行。Clang 驱动程序运行两个可执行文件:带有 -cc1 标志的 clang 可执行文件和链接器 ld 可执行文件}

过程可以如图 2.21 所示进行可视化,其中可以看到两个可执行文件作为编译过程的一部分被执行。第一个是带有特殊标志 -cc1 的 clang-18。第二个是链接器:ld。

\newpage

\mySubsubsection{2.3.4.}{综合所有步骤}

可以利用图 2.22 对所获得的知识进行总结。该图展示了由 Clang 驱动程序启动的两个不同的进程。第一个是 clang -cc1(编译器),第二个是 ld(链接器)。编译器进程与 Clang 驱动程序(clang)是同一个可执行文件,它是以一个特殊参数 -cc1 运行的。编译器产生一个目标文件,该文件随后被链接器(ld)处理,以生成最终的可执行文件。

\myGraphic{0.7}{content/part1/chapter2/images/22.png}{图 2.22: Clang 驱动程序:驱动程序接收输入文件 hello.cpp,启动了两个进程:clang 和 ld。第一个进程真正进行编译并启动集成汇编器。最后一个进程是链接器(ld),从编译器的结果和外部库(libstdc++)生成最终的可执行文件(image)}

图 2.22 中,可以看到之前提到的编译器的相似组成部分(参见第 2.2 节,开始了解编译器)。主要的区别在于预处理器(词法分析器的一部分)单独显示,而前端和中端则组合到了编译器中。此外,该图还描绘了一个由驱动程序执行汇编器,以生成目标代码。汇编器可以集成,如图 2.22 所示,也可能需要一个单独的进程来执行。

\begin{myNotic}{重要提示}
这里有一个使用 -c(仅编译)和 -o(输出文件)选项,以及使用平台标志来指定汇编器的例子:

\begin{shell}
$<...>/llvm-project/install/bin/clang -c hello.cpp \
                                      -o /tmp/hello.o
as -o /tmp/hello.o /tmp/hello.s
\end{shell}
\end{myNotic}

\mySubsubsection{2.3.5.}{调试 Clang}

我们将逐步讲解图2.14所示的编译过程的调试演示。

\begin{myNotic}{重要提示}
本书中的此调试演示和其他调试演示将使用第1.3.3节中构建的LLDB版本,即LLVM调试器及其构建和使用。读者们也可以使用宿主系统自带的LLDB。
\end{myNotic}

我们选择感兴趣的点,或者说是断点,是clang::ParseAST函数。在典型的调试演示中,类似于图1.11中概述的演示,在"-{}-"符号后输入命令行参数:

\begin{shell}
$ lldb <...>/llvm-project/install/bin/clang -- hello.cpp -o /tmp/hello \
                                               -lstdc++
\end{shell}

\begin{center}
图2.23:为hello.cpp文件编译运行调试器
\end{center}

这种情况下,<…>代表克隆LLVM项目时使用的目录路径。

不幸的是,这种方法对Clang编译器不起作用:

\begin{shell}
1  $ lldb <...>/llvm-project/install/bin/clang -- hello.cpp -o /tmp/hello.o -lstdc++
2  ...
3  (lldb) b clang::ParseAST
4  ...
5  (lldb) r
6  ...
7  2  locations added to breakpoint 1
8  ...
9  Process 247135 stopped and restarted: thread 1 received signal: SIGCHLD
10 Process 247135 stopped and restarted: thread 1 received signal: SIGCHLD
11 Process 247135 exited with status = 0 (0x00000000)
12 (lldb)
\end{shell}

\begin{center}
图2.24:失败的调试演示中断
\end{center}

从第7行可以看出,断点已经设置,但进程成功结束(第11行)而没有中断。换句话说,我们的断点在这个实例中没有触发。

理解Clang驱动程序的内部机制可以帮助我们处理手头的问题。clang可执行文件在这种情况下充当驱动程序,运行两个单独的进程(参考图2.21)。如果想要调试编译器,需要使用-cc1选项来运行。

\begin{myNotic}{重要提示}
值得一提的是,Clang在2019年实现了一项优化\footnote{Alexandre Ganea. [Clang][Driver] Re-use the calling process instead of creating a new process for the cc1 invocation. 2019. URL \url{https://reviews.llvm.org/D69825}}。当使用-c选项时,Clang驱动程序不会为编译器生成新进程:

\begin{shell}
$ <...>/llvm-project/install/bin/clang -c hello.cpp  \
                                       -o /tmp/hello.o \
                                       -###
clang version 18.1.0rc ...
InstalledDir: <...>/llvm-project/install/bin
(in-process)
"<...>/llvm-project/install/bin/clang-18" "-cc1"..."hello.cpp"
...
\end{shell}

Clang驱动程序没有生成新进程,而是在同一进程中调用"cc1"工具。这个特性不仅提高了编译器的性能,还可以用于Clang的调试。
\end{myNotic}

使用-cc1选项并排除-lstdc++选项(这是特定于第二个进程,即ld链接器的)后,调试器将生成以下输出:

\begin{shell}
1  $ lldb <...>/llvm-project/install/bin/clang -- -cc1 hello.cpp -o /tmp/hello.o
2  ...
3  (lldb) b clang::ParseAST
4  ...
5  (lldb) r
6  ...
7  2  locations added to breakpoint 1
8  Process 249890 stopped
9  * thread #1, name = 'clang', stop reason = breakpoint 1.1
10     frame #0: ... at ParseAST.cpp:117:3
11    114
12    115  void clang::ParseAST(Sema &S, bool PrintStats, bool SkipFunctionBodies) {
13    116    // Collect global stats on Decls/Stmts (until we have a module streamer).
14 -> 117    if (PrintStats) {
15    118      Decl::EnableStatistics();
16    119      Stmt::EnableStatistics();
17    120    }
18 (lldb) c
19 Process 249890 resuming
20 hello.cpp:1:10: fatal error: 'iostream' file not found
21 #include <iostream>
22          ^~~~~~~~~~
23 1  error generated.
24 Process 249890 exited with status = 1 (0x00000001)
25 (lldb)
\end{shell}

\begin{center}
图2.25:缺少搜索路径的调试演示
\end{center}

可以看到我们成功地设置了断点,但是进程以错误结束(见第20-24行)。这个错误是因为省略了一些搜索路径,这些路径通常是由Clang驱动程序隐式添加的,用于找到成功编译所需的所有包含文件。

如果显式地在编译器调用中包含所有必要的参数,就可以成功执行进程:

\begin{shell}
lldb <...>/llvm-project/install/bin/clang -- -cc1                    \
     -internal-isystem /usr/include/c++/13                           \
     -internal-isystem /usr/include/c++/13/x86_64-redhat-linux       \
     -internal-isystem <...>/llvm-project/install/lib/clang/18/include \
     -internal-externc-isystem /usr/include                          \
     hello.cpp                                                       \
     -o /tmp/hello.o
\end{shell}

\begin{center}
图2.26:指定搜索路径后运行调试器,系统是Fedora 39
\end{center}

可以为clang::ParseAST设置断点并运行调试器,执行将无错误完成:

\begin{shell}
1  (lldb) b clang::ParseAST
2  ...
3  (lldb) r
4  ...
5  2  locations added to breakpoint 1
6  Process 251736 stopped
7  * thread #1, name = 'clang', stop reason = breakpoint 1.1
8     frame #0: 0x00007fffe803eae0 ... at ParseAST.cpp:117:3
9     114
10    115  void clang::ParseAST(Sema &S, bool PrintStats, bool SkipFunctionBodies) {
11    116    // Collect global stats on Decls/Stmts (until we have a module streamer).
12 -> 117    if (PrintStats) {
13    118     Decl::EnableStatistics();
14    119     Stmt::EnableStatistics();
15    120    }
16 (lldb) c
17 Process 251736 resuming
18 Process 251736 exited with status = 0 (0x00000000)
19 (lldb)
\end{shell}

\begin{center}
图2.27:编译器的成功调试演示
\end{center}

总之,我们已经成功地演示了Clang编译器调用的调试。所展示的技术,可以有效地用于探索编译器的内部机制,并解决与编译器相关的错误。














