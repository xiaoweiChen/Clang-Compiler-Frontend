Clang编译器工具链遵循各种编译器书籍中广泛描述的模式\footnote{Alfred V. Aho, Monica S. Lam, Ravi Sethi, and Jeffrey D. Ullman. Compilers: Principles, Techniques, and Tools. Addison-Wesley, 2 edition, 2006. ISBN 978-0-321-48681-3}\footnote{Keith Cooper and Linda Torczon. Engineering A Compiler. Elsevier Inc., 2nd edition, 2012. ISBN 978-0-12-088478-0.},而Clang的前端部分与典型的编译器前端有很大的不同。这种区别的主要原因是C++语言的复杂性。一些特性,如宏,可以修改源代码本身,而其他特性,如typedef,可以影响token的类型。Clang还可以生成多种格式的输出。例如,以下命令会生成图2.5所示程序的悦目的HTML视图:

\begin{shell}
$ <...>/llvm-project/install/bin/clang -cc1 -emit-html max.cpp
\end{shell}

通过-cc1选项向Clang前端传递了生成源程序HTML形式的参数,也可以通过-Xclang选项向前端传递一个选项,这需要一个额外的参数来表示该选项本身:

\begin{shell}
$ <...>/llvm-project/install/bin/clang -Xclang -emit-html max.cpp \
                                       -fsyntax-only
\end{shell}

上述命令中,使用了-fsyntax-only选项,指示Clang只执行预处理器、解析器和语义分析阶段。

可以根据提供的编译选项,指导Clang前端执行不同的操作并产生不同类型的输出。这些操作的基础类称为\textbf{前端操作}(FrontendAction)。

\mySubsubsection{2.4.1.}{前端操作}

Clang前端一次只能执行一个前端操作。前端操作是基于提供的编译器选项,前端执行的具体任务或过程。以下是可能的前端操作的一些列表(表只包括可用前端操作的一个子集):

% Please add the following required packages to your document preamble:
% \usepackage{longtable}
% Note: It may be necessary to compile the document several times to get a multi-page table to line up properly
\begin{longtable}{|l|l|l|}
\hline
\textbf{前端操作} & \textbf{编译器选项} & \textbf{描述}                   \\ \hline
\endfirsthead
%
\endhead
%
EmitObjAction           & -emit-obj (default)      & 编译为目标文件              \\ \hline
EmitBCAction            & -emit-llvm-bc            & 编译成LLVM字节码               \\ \hline
EmitLLVMAction          & -emit-llvm               & 编译成LLVM可读形式          \\ \hline
ASTPrintAction          & -ast-print               & 构建AST,然后美观地打印它们。 \\ \hline
HTMLPrintAction         & -emit-html               & 以HTML形式打印程序源代码 \\ \hline
DumpTokensAction        & -dump-tokens             & 打印预处理器token             \\ \hline
\end{longtable}

\begin{center}
表2.1: 前端操作
\end{center}

\myGraphic{0.8}{content/part1/chapter2/images/28.png}{图2.28: Clang前端组件}

图2.28所示的图描绘了基本的前端架构,与图2.4所示的架构类似。Clang特定的差异是显而易见的。

一个显著的变化是对词法分析器的命名。在Clang中,词法分析器称为预处理器。这种命名约定表明了,词法分析器实现封装在Preprocessor类中。这种改变受到了C/C++语言独特特性的启发,这些特性包括需要特殊预处理(特殊类型)token(宏)。

虽然传统的编译器通常在解析器中同时执行语法和语义分析,但Clang将这些任务分布在不同的组件中。Parser组件专注于语法分析,而Sema组件处理语义分析。

此外,Clang提供了生成不同形式或格式的输出的能力。例如,CodeGenAction类作为各种代码生成操作的基础类,如EmitObjAction或EmitLLVMAction。

使用图2.5中的max函数的代码来探索Clang前端内部的细节:

\begin{cpp}
int max(int a, int b) {
  if (a > b)
    return a;
  return b;
}
\end{cpp}

\begin{center}
图2.29: max函数的源代码:max.cpp
\end{center}

通过使用-cc1选项,可以直接调用Clang前端,绕过驱动程序。这种方法能够更详细地检查和分析Clang前端的工作原理。

\mySubsubsection{2.4.2.}{预处理器}

第一部分是词法分析器,在Clang中称为预处理器。主要目标是将输入程序转换为token流,可以使用-dump-tokens选项打印token流:

\begin{shell}
$ <...>/llvm-project/install/bin/clang -cc1 -dump-tokens max.cpp
\end{shell}

输出如下:

\begin{shell}
int 'int'        [StartOfLine]  Loc=<max.cpp:1:1>
identifier 'max'         [LeadingSpace] Loc=<max.cpp:1:5>
l_paren '('             Loc=<max.cpp:1:8>
int 'int'               Loc=<max.cpp:1:9>
identifier 'a'   [LeadingSpace] Loc=<max.cpp:1:13>
comma ','               Loc=<max.cpp:1:14>
int 'int'        [LeadingSpace] Loc=<max.cpp:1:16>
identifier 'b'   [LeadingSpace] Loc=<max.cpp:1:20>
r_paren ')'             Loc=<max.cpp:1:21>
l_brace '{'      [LeadingSpace] Loc=<max.cpp:1:23>
if 'if'  [StartOfLine] [LeadingSpace]   Loc=<max.cpp:2:3>
l_paren '('      [LeadingSpace] Loc=<max.cpp:2:6>
identifier 'a'          Loc=<max.cpp:2:7>
greater '>'      [LeadingSpace] Loc=<max.cpp:2:9>
identifier 'b'   [LeadingSpace] Loc=<max.cpp:2:11>
r_paren ')'             Loc=<max.cpp:2:12>
return 'return'  [StartOfLine] [LeadingSpace]   Loc=<max.cpp:3:5>
identifier 'a'   [LeadingSpace] Loc=<max.cpp:3:12>
semi ';'                Loc=<max.cpp:3:13>
return 'return'  [StartOfLine] [LeadingSpace]   Loc=<max.cpp:4:3>
identifier 'b'   [LeadingSpace] Loc=<max.cpp:4:10>
semi ';'                Loc=<max.cpp:4:11>
r_brace '}'      [StartOfLine]  Loc=<max.cpp:5:1>
eof ''          Loc=<max.cpp:5:2>
\end{shell}

\begin{center}
图2.30: Clang打印token输出
\end{center}

有不同的token类型,如语言关键字(例如,int、return)、标识符(例如,max、a、b等)和特殊符号(例如,分号、逗号等)。这个小程序的token称为正常token,它们由词法分析器返回。

除了正常token外,Clang还有一种token类型,称为注释token。主要区别在于,这些token还存储额外的语义信息。例如,正常token可以解析器替换为一个包含关于类型或C++作用域信息的单个注释token。使用此类token的主要原因是为了性能,允许在解析器需要回退时防止重新解析。

由于注释token用于解析器的内部实现,假设使用LLDB的注释token示例:

\begin{cpp}
namespace clangbook {
template <typename T> class A {};
} // namespace clangbook
clangbook::A<int> a;
\end{cpp}

\begin{center}
图2.31: 使用注释token的源代码,annotation.cpp
\end{center}

代码的最后一行声明了变量a,其类型为: clangbook::A。该类型以注释token的形式表示,如下面的LLDB演示所示:

\begin{shell}
1  $ lldb <...>/llvm-project/install/bin/clang -- -cc1 annotation.cpp
2  ...
3  (lldb) b clang::Parser::ConsumeAnnotationToken
4  ...
5  (lldb) r
6  ...
7     608    }
8     609
9     610    SourceLocation ConsumeAnnotationToken() {
10 -> 611      assert(Tok.isAnnotation() && "wrong consume method");
11    612      SourceLocation Loc = Tok.getLocation();
12    613      PrevTokLocation = Tok.getAnnotationEndLoc();
13    614      PP.Lex(Tok);
14 (lldb) p Tok.getAnnotationRange().printToString(PP.getSourceManager())
15 (std::string) "<annotation.cpp:4:1, col:17>"
\end{shell}

\begin{center}
图2.32: annotation.cpp的LLDB交互
\end{center}

Clang从图2.31中示例程序的第4行消耗了一个注释token。该token位于列1和列7之间。这对应于以下作为token使用的文本:clangbook::A。该token由其他token组成,例如'clangbook'、'::'等。将所有token组合成一个简化解析,并提高整体的解析性能。

C/C++语言有一些特定之处,影响了Preprocessor类的内部实现。首先是宏。Preprocessor类有两个不同的辅助类来检索token:

\begin{itemize}
\item
Lexer类用于将文本缓冲区转换为token流。

\item
TokenLexer类用于从宏扩展中检索token。
\end{itemize}

这些辅助类中只能有一个同时激活。

C/C++的另一个特定之处是\#include指令(也适用于import指令),需要维护一个包含的栈,其中每个包含都可以有自己的TokenLexer或Lexer,取决于其中是否包含宏扩展。Preprocessor类为每个\#include指令保持了一个包含的lexer栈(IncludeMacroStack类),如图2.33所示。

\myGraphic{0.8}{content/part1/chapter2/images/33.png}{图2.33: 预处理器(Clang词法分析器)类内部}

\mySubsubsection{2.4.2.}{解析器与语义分析器}

解析器和语义分析器是Clang编译器前端的关键组件,处理源代码的语法和语义分析,产生AST作为输出。对于测试程序,可以使用以下命令可视化这个树:

\begin{shell}
$ <...>/llvm-project/install/bin/clang -cc1 -ast-dump max.cpp
\end{shell}

输出如下:

\begin{shell}
TranslationUnitDecl 0xa9cb38 <<invalid sloc>> <invalid sloc>
|-TypedefDecl 0xa9d3a8 <<invalid sloc>> <invalid sloc>
implicit __int128_t '__int128'
| '-BuiltinType 0xa9d100 '__int128'
...
'-FunctionDecl 0xae6a98 <max.cpp:1:1, line:5:1> line:1:5 max
'int (int, int)'
  |-ParmVarDecl 0xae6930 <col:9, col:13> col:13 used a 'int'
  |-ParmVarDecl 0xae69b0 <col:16, col:20> col:20 used b 'int'
  '-CompoundStmt 0xae6cd8 <col:23, line:5:1>
    |-IfStmt 0xae6c70 <line:2:3, line:3:12>
    | |-BinaryOperator 0xae6c08 <line:2:7, col:11> 'bool' '>'
    | | |-ImplicitCastExpr 0xae6bd8 <col:7> 'int' <LValueToRValue>
    | | | '-DeclRefExpr 0xae6b98 <col:7> 'int' lvalue ParmVar 0xae6930
            'a' 'int'
    | | '-ImplicitCastExpr 0xae6bf0 <col:11> 'int' <LValueToRValue>
    | |   '-DeclRefExpr 0xae6bb8 <col:11> 'int' lvalue ParmVar 0xae69b0
            'b' 'int'
    | '-ReturnStmt 0xae6c60 <line:3:5, col:12>
    |   '-ImplicitCastExpr 0xae6c48 <col:12> 'int' <LValueToRValue>
    |     '-DeclRefExpr 0xae6c28 <col:12> 'int' lvalue ParmVar 0xae6930
            'a' 'int'
    '-ReturnStmt 0xae6cc8 <line:4:3, col:10>
      '-ImplicitCastExpr 0xae6cb0 <col:10> 'int' <LValueToRValue>
        '-DeclRefExpr 0xae6c90 <col:10> 'int' lvalue ParmVar 0xae69b0
            'b' 'int'
\end{shell}

\begin{center}
图2.34: Clang AST dump的输出
\end{center}

Clang使用了一个手工编写的递归下降解析器\footnote{LLVM Community. Clang features. 2023. URL \url{https://clang.llvm.org/features.html}}。这个解析器很简单,这也是选中它的一个关键原因。此外,C/C++语言的复杂规则需要一个易于适应的解析器。

通过示例来探索它是如何工作的。解析从代表单个翻译单元的顶级声明开始,称为TranslationUnitDecl。

一个源文件及其通过预处理指令\#include包含的所有头文件(16.5.1.2)和源文件,减去通过条件包含预处理指令(15.2)跳过的源行,称为翻译单元。

解析器首先识别出源代码的初始token对应于函数定义,如C++标准所定义\footnote{International Organization for Standardization. International Standard ISO/IEC 14882:2020(E) – Programming Languages – C++. International Organization for Standardization, 2020. URL \url{https://www.iso.org/standard/73560.html}, dcl.fct.def.general}:

\begin{shell}
function-definition :
    ... declarator ... function-body
    ...
\end{shell}

\begin{center}
图2.35: C++标准中的函数定义
\end{center}

相应的代码如下:

\begin{cpp}
int max(...) {
  ...
}
\end{cpp}

\begin{center}
图2.36: 对应于C++标准中函数定义的部分示例代码
\end{center}

函数定义需要一个声明器和函数体。首先从声明器开始:

\begin{shell}
declarator:
        ...
        ... parameters-and-qualifiers ...
...
parameters-and-qualifiers:
        ( parameter-declaration-clause ) ...
...
parameter-declaration-clause:
        parameter-declaration-list ...
parameter-declaration-list:
        parameter-declaration
        parameter-declaration-list , parameter-declaration
\end{shell}

\begin{center}
图2.37: C++标准中的声明器定义
\end{center}

声明器在一个括号内指定参数声明列表,源码中的相应部分如下所示:

\begin{cpp}
... (int a, int b)
...
\end{cpp}

\begin{center}
图2.38: 对应于C++标准中声明器的部分示例代码
\end{center}

函数定义,还需要一个函数体。C++标准将函数体定义为:

\begin{shell}
function-body:
    ... compound-statement
    ...
\end{shell}

\begin{center}
图2.39: C++标准中的函数体定义
\end{center}

函数体由一个复合语句组成,C++标准定义为:

\begin{shell}
compound-statement:
    { statement-seq ... }
statement-seq:
    statement
    statement-seq statement
\end{shell}

\begin{center}
图2.40: C++标准中的复合语句定义
\end{center}

其描述了一组语句,这些语句都括在大括号内(\{... ).

我们的程序有两种类型的语句:条件(if)语句和返回语句。这些在C++语法定义中如下所示:

\begin{shell}
statement:
    ...
    selection-statement
    ...
    jump-statement
    ...
\end{shell}

\begin{center}
图2.41: C++标准中的语句定义
\end{center}

选择语句对应于我们程序中的if条件,而跳转语句对应于返回运算符。

更详细地检查跳转语句:

\begin{shell}
jump-statement:
    ...
    return expr-or-braced-init-list;
    ...
\end{shell}

\begin{center}
图2.42: C++标准中的跳转语句定义
\end{center}

其中expr-or-braced-init-list定义为:

\begin{shell}
expr-or-braced-init-list:
    expression
    ...
\end{shell}

\begin{center}
图2.43: C++标准中的返回表达式定义
\end{center}

return关键字后面跟着一个表达式和一个分号,其中有一个隐式转换表达式,自动将变量转换为所需的类型(int)。

通过LLDB调试器检查解析器的操作可能会很有启发:

\begin{shell}
$ lldb <...>/llvm-project/install/bin/clang -- -cc1 max.cpp
\end{shell}

调试器演示输出在图2.44中。在第1行,为返回语句的解析设置了断点。程序有两个返回语句。跳过了第一个调用(第4行),并在第二个方法调用(第9行)处暂停。调用栈(从第13行的'bt'命令)显示了解析过程的调用栈。这个栈反映了之前描述的解析块,遵循了C++语法详细说明。

\begin{shell}
1  (lldb) b clang::Parser::ParseReturnStatement
2  (lldb) r
3  ...
4  (lldb) c
5  ...
6  * thread #1, name = 'clang', stop reason = breakpoint 1.1
7     frame #0: ... clang::Parser::ParseReturnStatement(...) ...
8     2421 StmtResult Parser::ParseReturnStatement() {
9  -> 2422   assert((Tok.is(tok::kw_return) || Tok.is(tok::kw_co_return)) &&
10    2423         "Not a return stmt!");
11    2424   bool IsCoreturn = Tok.is(tok::kw_co_return);
12    2425   SourceLocation ReturnLoc = ConsumeToken();  // eat the 'return'.
13 (lldb) bt
14   * frame #0: ... clang::Parser::ParseReturnStatement( ...
15    ...
16    frame #2: ... clang::Parser::ParseStatementOrDeclaration( ...
17    frame #3: ... clang::Parser::ParseCompoundStatementBody( ...
18    frame #4: ... clang::Parser::ParseFunctionStatementBody( ...
19    frame #5: ... clang::Parser::ParseFunctionDefinition( ...
20    ...
\end{shell}

\begin{center}
图2.44: 在max.cpp示例程序中解析第二个返回语句
\end{center}

解析的结果是生成AST,也可以使用调试器检查AST的创建过程。为此,需要在clang::ReturnStmt::Create上设置断点:

\begin{shell}
1  $ lldb <...>/llvm-project/install/bin/clang -- -cc1 max.cpp
2  ...
3  (lldb) b clang::ReturnStmt::Create
4  (lldb) r
5  ...
6  (lldb) c
7  ...
8  * thread #1, name = 'clang', stop reason = breakpoint 1.1
9     frame #0: ... clang::ReturnStmt::Create(...) at Stmt.cpp:1205:8
10    1202
11    1203 ReturnStmt *ReturnStmt::Create(const ASTContext &Ctx, SourceLocation RL,
12    1204                              Expr *E, const VarDecl *NRVOCandidate) {
13 -> 1205   bool HasNRVOCandidate = NRVOCandidate != nullptr;
14    1206   ...
15    1207   ...
16    1208   return new (Mem) ReturnStmt(RL, E, NRVOCandidate);
17 (lldb) bt
18 * thread #1, name = 'clang', stop reason = breakpoint 1.1
19  * frame #0: ... clang::ReturnStmt::Create( ...
20    frame #1: ... clang::Sema::BuildReturnStmt( ...
21    frame #2: ... clang::Sema::ActOnReturnStmt( ...
22    frame #3: ... clang::Parser::ParseReturnStatement( ...
23    frame #4: ... clang::Parser::ParseStatementOrDeclarationAfterAttributes( ...
24    ...
\end{shell}

\begin{center}
图2.45: 在clang::ReturnStmt::Create处的断点
\end{center}

返回语句的AST节点由Sema组件创建。

返回语句解析的开始可以在frame \#4找到:

\begin{shell}
1  (lldb) f 4
2  frame #4: ... clang::Parser::ParseStatementOrDeclarationAfterAttributes( ...
3     323     SemiError = "break";
4     324     break;
5     325    case tok::kw_return:             // C99 6.8.6.4: return-statement
6  -> 326     Res = ParseReturnStatement();
7     327     SemiError = "return";
8     328     break;
9     329    case tok::kw_co_return:           // C++ Coroutines: ...
10  (lldb)
\end{shell}

\begin{center}
图2.46: 调试器中的返回语句解析
\end{center}

这里,有对C99标准\footnote{International Organization for Standardization (ISO). ISO/IEC 9899:1999 - Programming languages - C. International Organization for Standardization (ISO), 1999. URL \url{https://www.iso.org/standard/23482.html}。} 的引用,该标准为相应的语句提供了详细的描述。

代码假设当前token的类型为tok::kw\_return,解析器调用clang::Parser::ParseReturnStatement方法。

不同C++构造的AST节点创建过程可能会有所不同,但一般来说,都会遵循图2.47所示的模式。

\myGraphic{0.8}{content/part1/chapter2/images/47.png}{图2.47: Clang前端中的C++解析}

图2.47中,方框代表相应的类,函数调用用带有调用函数的边缘表示。解析器调用Preprocessor::Lex方法从词法分析器获取一个token,然后调用与token对应的函数,例如Parser::ParseXXX对于token XXX。然后该方法调用Sema::ActOnXXX,使用XXX::Create创建相应的对象。这个过程后,用新的token重复。

现在已经全面探索了Clang中,典型的编译器前端流程的实现。词法分析器组件(预处理器)与解析器(包括解析器和语义分析器)协同工作,生成了未来代码生成的主要数据结构:抽象语法树(AST)。AST不仅对于代码生成至关重要,对于代码分析和修改也非常重要。Clang提供了对AST的便捷访问,从而使得开发各种编译器工具成为可能。












