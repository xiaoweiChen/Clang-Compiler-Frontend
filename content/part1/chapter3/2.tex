AST通常表示为一棵树,其叶节点对应于各种对象,例如函数声明和循环体。通常,AST代表了语法分析的结果,即解析。Clang的AST节点设计为不可变。这种设计要求Clang AST存储语义分析的结果,所以Clang AST代表了语法和语义分析的结果。

\begin{myNotic}{重要提示}
虽然Clang也使用AST,但Clang AST并不是真正的树。由于存在反向边,因此"图"更适合用来描述Clang的AST
\end{myNotic}

C++中实现的典型树结构将所有节点从基类派生。Clang采用了不同的方法,将不同的C++构造划分为不同的组,并为每个组提供基本类:

\begin{itemize}
\item
语句:clang::Stmt是所有语句的基本类。包括普通语句,如if语句(clang::IfStmt类),以及表达式和其他C++构造。

\item
声明:clang::Decl是声明的基本类。这包括变量、typedef、函数、结构体等。还有单独的基类用于具有上下文的声明,即可能包含其他声明的声明。该类称为clang::DeclContext。clang::DeclContext中包含的声明可以通过clang::DeclContext::decls方法访问。翻译单元(clang::TranslationUnitDecl类)和命名空间(clang::NamespaceDecl类)是具有上下文的声明的典型示例。

\item
类型:C++有一个丰富的类型系统。包括基本类型,如整数int,以及自定义定义的类型和通过typedef或using重定义的类型。C++的类型可以具有const等修饰符,可以表示不同的内存寻址模式,即指针、引用等。Clang使用clang::Type作为AST中类型表示的基本类。
\end{itemize}

这些组之间还有其他关系。例如,继承自clang::Stmt的clang::DeclStmt类具有检索相应声明的方法。表达式(由clang::Expr类表示),继承自clang::Stmt,具有与类型一起工作的方法。

\mySubsubsection{3.2.1.}{语句}

Stmt是所有语句的基本类,语句可以分为两组(见图3.1)。第一组包含有值的语句,而相反的组包含无值的语句。

\myGraphic{1.0}{content/part1/chapter3/images/1.png}{图3.1:Clang AST:语句}

无值语句组包含各种C++构造,如if语句(clang::IfStmt类)或复合语句(clang::CompoundStmt类)。大多数语句都属于这个组。

有值语句组包含一个基类clang::ValueStmt,有几个子类,如clang::LabelStmt(用于标签表示)或clang::ExprStmt(用于表达式表示),见图3.2。

\myGraphic{1.0}{content/part1/chapter3/images/2.png}{图3.2:Clang AST:有值的语句}

\mySubsubsection{3.2.2.}{声明}

声明也可以分为两大类:带上下文的声明和不带上下文的声明。带上下文的声明可以被视为其他声明的占位符。例如,一个C++命名空间、翻译单元或函数声明可能包含其他声明。一个朋友实体声明(clang::DeclFriend)可以视为一个不带上下文的声明的例子。

需要注意的是,继承自DeclContext的类也有clang::Decl作为其顶层父类。

一些声明可以重新声明:

\begin{cpp}
extern int a;
int a = 1;
\end{cpp}

\begin{center}
图3.3:声明示例:redeclaration.cpp
\end{center}

这样的声明有一个父类,通过clang::Redeclarable<…>模板实现。

\mySubsubsection{3.2.3.}{类型}

C++是一种静态类型语言,所以变量的类型必须在编译时声明。类型允许编译器对程序的含义做出合理的推断,这是语义分析的重要组成部分。clang::Type是Clang中类型的基本类。

C/C++的类型可能有修饰符,称为CV修饰符,如标准\footnote{International Organization for Standardization. International Standard ISO/IEC 14882:2020(E) – Programming Languages – C++. International Organization for Standardization, 2020. URL \url{https://www.iso.org/standard/73560.html}}, basic.type.qualifier]所规定。CV在这里代表两个关键字const和volatile,可以作为类型的修饰符。

\begin{myNotic}{重要提示}
C99标准有一个类型修饰符——restrict,Clang也支持[\footnote{International Organization for Standardization (ISO). ISO/IEC 9899:1999 - Programming languages - C. International Organization for Standardization (ISO), 1999. URL \url{https://www.iso.org/standard/23482.html}}, 6.7.3]。这个类型修饰符指示编译器,在指针的生命周期内,不会使用其他指针来访问指向的对象。这使得编译器可以执行无法优化,如向量化。restrict有助于限制指针别名效应,即当多个指针引用同一内存位置时发生的情况,从而有助于优化。如果程序员的意图没有遵循,并且对象使用一个独立的指针访问,这会导致未定义的行为。
\end{myNotic}

Clang有一个特殊的类来支持带有修饰符的类型,clang::QualType,是一个指向clang::Type的指针和包含类型修饰符信息的位掩码的组合。这个类有一个方法来获取指向clang::Type的指针并检查不同的修饰符。以下代码(LLVM 18.x, clang/lib/AST/ExprConstant.cpp, 第3918行)展示了如何检查类型是否有const修饰符:

\begin{cpp}
bool checkConst(QualType QT) {
  // Assigning to a const object has undefined behavior.
  if (QT.isConstQualified()) {
    Info.FFDiag(E, diag::note_constexpr_modify_const_type) << QT;
    return false;
  }
  return true;
}
\end{cpp}

\begin{center}
图3.4:clang/lib/AST/ExprConstant.cpp中的checkConst实现
\end{center}

clang::QualType实现了operator->()和operator*(),可以视为指向clang::Type类的智能指针。

除了修饰符之外,类型还可以有其他信息,表示不同的内存地址模型。例如,可以有一个指向对象的指针或引用。clang::Type有如下辅助方法来检查不同的地址模型:

\begin{itemize}
\item
clang::Type::isPointerType()于检查指针类型

\item
clang::Type::isReferenceType()用于检查引用类型
\end{itemize}

C/C++的类型也可以使用别名,通过使用typedef或using关键字引入。以下代码定义了foo和bar作为int类型的别名。

\begin{cpp}
using foo = int;
typedef int bar;
\end{cpp}

\begin{center}
图3.5:类型别名声明
\end{center}

原始类型,在例子中是int,称为规范类型。可以使用clang::QualType::isCanonical()方法来测试类型是否是规范类型。clang::QualType还提供了一个方法来从别名获取规范类型:clang::QualType::getCanonicalType()。

了解了Clang中用于AST的基本块之后,现在是时候研究这些块如何用于AST遍历,这是编译器和编译工具的基本操作。









































