编译器需要遍历AST来生成IR代码。因此,为树遍历提供良好结构的数据结构对于AST设计至关重要。换句话说,AST的设计应该优先考虑促进容易的树遍历。许多系统中的标准方法是为所有AST节点提供一个共同的基类。这个类通常提供一种方法来检索节点的子节点,从而允许使用诸如广度优先搜索(BFS)[19]之类的流行算法进行树遍历。然而,Clang采取了不同的方法:它的AST节点没有共同的祖先。这提出了一个问题:Clang中的树遍历是如何组织的?

Clang采用了三种独特的技术:

\begin{itemize}
\item
用于访问者类定义的奇异递归模板模式(CRTP)

\item
为不同节点量身定制的特有方法

\item
宏,可以被视为特有方法与CRTP之间的连接层
\end{itemize}

我们将通过一个简单的程序来探索这些技术,该程序旨在识别函数定义并显示函数名及其参数。

\mySubsubsection{3.3.1.}{DeclVisitor测试工具}

我们的测试工具将建立在clang::DeclVisitor类之上,这是一个简单的访问者类,用于帮助创建C/C++声明的访问者。

我们将使用与我们的第一个Clang工具相同的CMake文件(参见图1.13)。新工具的唯一添加是clangAST库。生成的CMakeLists.txt如图3.6所示:

\begin{cmake}
cmake_minimum_required(VERSION 3.16)
project("declvisitor")

if ( NOT DEFINED ENV{LLVM_HOME})
  message(FATAL_ERROR "$LLVM_HOME is not defined")
else()
  message(STATUS "$LLVM_HOME found: $ENV{LLVM_HOME}")
  set(LLVM_HOME $ENV{LLVM_HOME} CACHE PATH "Root of LLVM installation")
  set(LLVM_LIB ${LLVM_HOME}/lib)
  set(LLVM_DIR ${LLVM_LIB}/cmake/llvm)
  find_package(LLVM REQUIRED CONFIG)
  include_directories(${LLVM_INCLUDE_DIRS})
  link_directories(${LLVM_LIBRARY_DIRS})
  set(SOURCE_FILE DeclVisitor.cpp)
  add_executable(declvisitor ${SOURCE_FILE})
  set_target_properties(declvisitor PROPERTIES COMPILE_FLAGS "-fno-rtti")
  target_link_libraries(declvisitor
    LLVMSupport
    clangAST
    clangBasic
    clangFrontend
    clangSerialization
    clangTooling
   )
\end{cmake}

\begin{center}
图3.6:DeclVisitor测试工具的CMakeLists.txt文件
\end{center}

工具的main函数如下所示:

\begin{cpp}
#include "clang/Tooling/CommonOptionsParser.h"
#include "clang/Tooling/Tooling.h"
#include "llvm/Support/CommandLine.h" // llvm::cl::extrahelp

#include "FrontendAction.hpp"

namespace {
  llvm::cl::OptionCategory TestCategory("Test project");
  llvm::cl::extrahelp
    CommonHelp(clang::tooling::CommonOptionsParser::HelpMessage);
} // namespace

int main(int argc, const char **argv) {
  llvm::Expected<clang::tooling::CommonOptionsParser> OptionsParser =
    clang::tooling::CommonOptionsParser::create(argc, argv, TestCategory);
  if (!OptionsParser) {
    llvm::errs() << OptionsParser.takeError();
    return 1;
  }
  clang::tooling::ClangTool Tool(OptionsParser->getCompilations(),
                                 OptionsParser->getSourcePathList());
  return Tool.run(clang::tooling::newFrontendActionFactory<
                       clangbook::declvisitor::FrontendAction>()
                       .get());
}
\end{cpp}

\begin{center}
图3.7: DeclVisitor测试工具的main函数
\end{center}

从第5行和第23行可以看出,我们使用了针对我们项目的自定义前端动作:clangbook::declvisitor::FrontendAction。

以下是对这个类的代码:

\begin{cpp}
#include "Consumer.hpp"
#include "clang/Frontend/FrontendActions.h"

namespace clangbook {
namespace declvisitor {
class FrontendAction : public clang::ASTFrontendAction {
public:
  virtual std::unique_ptr<clang::ASTConsumer>
  CreateASTConsumer(clang::CompilerInstance &CI,
                    llvm::StringRef File) override {
    return std::make_unique<Consumer>();
  }
};
} // namespace declvisitor
}// namespace clangbook
\end{cpp}

\begin{center}
 图3.8: DeclVisitor测试工具的自定义FrontendAction类
\end{center}

你会注意到,我们已经重写了clang::ASTFrontendAction类的CreateASTConsumer函数,以实例化我们自定义的AST Consumer类,该类定义在clangbook::declvisitor命名空间中,如图3.8。

该类的实现如下:

\begin{cpp}
#include "Visitor.hpp"
#include "clang/Frontend/ASTConsumers.h"

namespace clangbook {
namespace declvisitor {
class Consumer : public clang::ASTConsumer {
public:
  Consumer() : V(std::make_unique<Visitor>()) {}

  virtual void HandleTranslationUnit(clang::ASTContext &Context) override {
    V->Visit(Context.getTranslationUnitDecl());
  }

private:
  std::unique_ptr<Visitor> V;
};
} // namespace declvisitor
} // namespace clangbook
\end{cpp}

\begin{center}
图3.9: DeclVisitor测试工具的Consumer类
\end{center}

在这里,我们可以看到我们已经创建了一个示例访问者,并使用重写的clang::ASTConsumer类的HandleTranslationUnit方法(见图3.9,第11行)来调用它。

然而,最有趣的部分是Visitor代码:

\begin{cpp}
#include "clang/AST/DeclVisitor.h"

namespace clangbook {
namespace declvisitor {
class Visitor : public clang::DeclVisitor<Visitor> {
public:
  void VisitFunctionDecl(const clang::FunctionDecl *FD) {
    llvm::outs() << "Function: '" << FD->getName() << "'\n";
    for (auto Param : FD->parameters()) {
      Visit(Param);
    }
  }
  void VisitParmVarDecl(const clang::ParmVarDecl *PVD) {
    llvm::outs() << "\tParameter: '" << PVD->getName() << "'\n";
  }
  void VisitTranslationUnitDecl(const clang::TranslationUnitDecl *TU) {
    for (auto Decl : TU->decls()) {
      Visit(Decl);
    }
  }
};
} // namespace declvisitor
} // namespace clangbook
\end{cpp}

\begin{center}
图3.10: Visitor类的实现
\end{center}

我们将在稍后更深入地探讨代码。目前,我们观察到它在第8行打印了函数名,在第14行打印了参数名。

我们可以使用与我们测试项目相同的命令序列来编译我们的程序,如第1.4节,测试项目——使用Clang工具进行语法检查中所述。

\begin{shell}
export LLVM_HOME=<...>/llvm-project/install
mkdir build
cd build
cmake -G Ninja -DCMAKE_BUILD_TYPE=Debug ...
ninja
\end{shell}

\begin{center}
图3.11: DeclVisitor测试工具的配置和构建命令
\end{center}

正如你所注意到的,我们在CMake中使用了-DCMAKE\_BUILD\_TYPE=Debug选项。我们之所以使用这个选项,是因为我们可能需要在调试器中调查结果程序。

\begin{myNotic}{重要提示}
我们为工具使用的构建命令假设所需的库安装在<…>/llvm-project/install文件夹下,这是在CMake配置命令期间使用-DCMAKE\_INSTALL\_PREFIX选项指定的,如第1.4节,测试项目——使用Clang工具进行语法检查中所述。见图1.12:

\begin{shell}
cmake -G Ninja -DCMAKE_BUILD_TYPE=Debug -DCMAKE_INSTALL_PREFIX=../install -DLLVM_TARGETS_TO_BUILD="X86" -DLLVM_ENABLE_PROJECTS="clang" -DLLVM_USE_LINKER=gold -DLLVM_USE_SPLIT_DWARF=ON -DBUILD_SHARED_LIBS=ON ../llvm
\end{shell}

所需的构建工件必须使用ninja install命令安装
\end{myNotic}

我们将使用我们在之前调查中引用的同一程序(见图2.5)来研究AST遍历:

\begin{cpp}
int max(int a, int b) {
  if (a > b)
    return a;
  return b;
}
\end{cpp}

\begin{center}
图3.12: 测试程序max.cpp
\end{center}

这个程序包含一个单一的函数max,它接受两个参数a和b,并返回两者中的较大者。

我们可以像这样运行我们的程序:

\begin{shell}
$ ./declvisitor max.cpp -- -std=c++17
...
Function: 'max'
        Parameter: 'a'
        Parameter: 'b'
\end{shell}

图3.13: 在测试文件上运行declvisitor实用程序的结果

\begin{myNotic}{重要提示}

我们在图3.13中使用'-{}-'来向编译器传递附加参数,具体指示我们想要使用C++17,并通过选项'-std=c++17'。我们还可以传递其他编译器参数。另一种方法是使用'-p'选项指定编译数据库路径,如下所示:

\begin{shell}
$ ./declvisitor max.cpp -p <path>
\end{shell}

在这里,是包含编译数据库的文件夹路径<path>。关于编译数据库的更多信息,请参见第9章,附录1:编译数据库。

\end{myNotic}

现在,让我们详细探讨访问者类的实现。

\mySubsubsection{3.3.2.}{访问者实现}

让我们深入研究访问者代码(见图3.10)。首先,你会注意到我们的类是从一个基类参数化而成的,这个基类是我们的类:

\begin{cpp}
class Visitor : public clang::DeclVisitor<Visitor> {
\end{cpp}

\begin{center}
图3.14: 访问者类声明
\end{center}

这种构造被称为奇异递归模板模式,或CRTP。

访问者类有几个回调,当遇到相应的AST节点时会被触发。第一个回调针对表示函数声明的AST节点:

\begin{cpp}
void VisitFunctionDecl(const clang::FunctionDecl *FD) {
  llvm::outs() << "Function: '" << FD->getName() << "'\n";
  for (auto Param : FD->parameters()) {
    Visit(Param);
  }
}
\end{cpp}

\begin{center}
图3.15: FunctionDecl回调
\end{center}

如图3.15所示,函数名在第8行被打印出来。我们的下一步是打印参数名。为了检索函数参数,我们可以使用clang::FunctionDecl类的parameters()方法。这个方法之前被提及为AST遍历的特有方法。每个AST节点都提供自己的方法来访问子节点。由于我们有一个特定类型的AST节点(即clang::FunctionDecl*)作为参数,我们可以使用这些方法。

函数参数被传递给基类clang::DeclVisitor<>的Visit(…)方法,如图3.15第12行所示。这个调用随后被转换为另一个回调,具体针对clang::ParmVarDecl AST节点:

\begin{cpp}
void VisitParmVarDecl(const clang::ParmVarDecl *PVD) {
  llvm::outs() << "\tParameter: '" << PVD->getName() << "'\n";
}
\end{cpp}

\begin{center}
图3.16: ParmVarDecl回调
\end{center}

你可能想知道这个转换是如何实现的。答案在于CRTP和C/C++宏的结合。为了理解这一点,我们需要深入研究clang::DeclVisitor<>类的Visit()方法的实现。这个实现严重依赖于C/C++宏,因此为了看到实际的代码,我们必须展开这些宏。这可以通过使用shell-E编译器选项来实现。让我们对CMakeLists.txt进行一些修改,并引入一个新的自定义目标。

\begin{cmake}
add_custom_command(
  OUTPUT ${SOURCE_FILE}.preprocessed
  COMMAND ${CMAKE_CXX_COMPILER} -E -I ${LLVM_HOME}/include ${CMAKE_CURRENT_SOURCE_DIR}/${SOURCE_FILE} > ${SOURCE_FILE}.preprocessed
  DEPENDS ${SOURCE_FILE}
  COMMENT "Preprocessing ${SOURCE_FILE}"
)
add_custom_target(preprocess ALL DEPENDS ${SOURCE_FILE}.preprocessed)
\end{cmake}

\begin{center}
图3.17: 展开宏的自定义目标
\end{center}

我们可以像这样运行目标:

\begin{shell}
$ ninja preprocess
\end{shell}

生成的文件可以在之前指定的构建文件夹中找到,名为DeclVisitor.cpp.preprocessed。我们之前在执行cmake命令时指定了包含文件的构建文件夹(见图3.11)。在这个文件中,Visit()方法的生成代码如下所示:

\begin{cpp}
RetTy Visit(typename Ptr<Decl>::type D) {
  switch (D->getKind()) {
  ...
  case Decl::ParmVar: return static_cast<ImplClass*>(this)->VisitParmVarDecl(static_cast<typename Ptr<ParmVarDecl>::type>(D));
  ...
  }
}
\end{cpp}

\begin{center}
图3.18: Visit()方法的生成代码
\end{center}

这段代码展示了Clang中CRTP的使用。在这个上下文中,CRTP被用来将调用重定向回我们的Visitor类,这个类被称为ImplClass。CRTP允许基类调用继承类的某个方法。这种模式可以作为虚拟函数的替代方案,并提供了几个优点,其中最显著的是与性能相关的优点。具体来说,方法调用是在编译时解决的,消除了与虚拟方法调用相关的vtable查找的需要。

代码是通过C/C++宏生成的,正如这里所展示的。这段特定的代码来源于clang/include/clang/AST/DeclVisitor.h头文件:

\begin{cpp}
#define DISPATCH(NAME, CLASS) \
  return static_cast<ImplClass*>(this)->Visit##NAME(static_cast<PTR(CLASS)>(D))
\end{cpp}

\begin{center}
图3.19: clang/include/clang/AST/DeclVisitor.h中的DISPATCH宏定义
\end{center}

图3.19中的NAME被替换为节点名称;在我们的例子中,它是ParmVarDecl。

DeclVisitor用于遍历C++声明。Clang还提供了StmtVisitor和TypeVisitor来分别遍历语句和类型。这些访问者是基于我们与声明访问者示例考虑的相同原则构建的。然而,这些访问者存在一些问题。它们只能与特定组的AST节点一起使用。例如,DeclVisitor只能与Decl类的后代一起使用。另一个限制是我们需要实现递归。例如,我们在图3.10的第9-11行设置了递归来遍历函数声明。同样,我们使用了递归来遍历翻译单元内的声明(见图3.10,第17-19行)。这带来了一个新的问题:我们可能会错过递归的某些部分。例如,如果max函数声明指定在命名空间内,我们的代码将无法正确工作。为了解决这种情况,我们需要为命名空间声明实现一个额外的访问方法。

这些问题通过递归访问者得到了解决,我们将在不久的将来讨论它。


















