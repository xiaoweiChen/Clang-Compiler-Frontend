递归AST访问者解决了与专用访问者观察到的局限性。创建相同的程序,该程序搜索并打印函数声明及其参数,但这次我们将使用递归访问者。

递归访问者测试工具的CMakeLists.txt将以与之前类似的方式使用,只有项目名称(图3.20第2行和第15-17行)和源文件名(图3.20第14行)发生了变化:

\begin{cmake}
cmake_minimum_required(VERSION 3.16)
project("recursivevisitor")

if ( NOT DEFINED ENV{LLVM_HOME})
  message(FATAL_ERROR "$LLVM_HOME is not defined")
else()
  message(STATUS "$LLVM_HOME found: $ENV{LLVM_HOME}")
  set(LLVM_HOME $ENV{LLVM_HOME} CACHE PATH "Root of LLVM installation")
  set(LLVM_LIB ${LLVM_HOME}/lib)
  set(LLVM_DIR ${LLVM_LIB}/cmake/llvm)
  find_package(LLVM REQUIRED CONFIG)
  include_directories(${LLVM_INCLUDE_DIRS})
  link_directories(${LLVM_LIBRARY_DIRS})
  set(SOURCE_FILE RecursiveVisitor.cpp)
  add_executable(recursivevisitor ${SOURCE_FILE})
  set_target_properties(recursivevisitor PROPERTIES COMPILE_FLAGS "-fno-rtti")
  target_link_libraries(recursivevisitor
    LLVMSupport
    clangAST
    clangBasic
    clangFrontend
    clangSerialization
    clangTooling
  )
endif()
\end{cmake}

\begin{center}
图3.20: 递归访问者测试工具的CMakeLists.txt文件
\end{center}

工具的主函数与图3.7中定义的'DeclVisitor'主函数类似。

\begin{cpp}
#include "clang/Tooling/CommonOptionsParser.h"
#include "clang/Tooling/Tooling.h"
#include "llvm/Support/CommandLine.h" // llvm::cl::extrahelp

#include "FrontendAction.hpp"

namespace {
llvm::cl::OptionCategory TestCategory("Test project");
llvm::cl::extrahelp
  CommonHelp(clang::tooling::CommonOptionsParser::HelpMessage);
} // namespace

int main(int argc, const char **argv) {
  llvm::Expected<clang::tooling::CommonOptionsParser> OptionsParser =
    clang::tooling::CommonOptionsParser::create(argc, argv, TestCategory);
  if (!OptionsParser) {
    llvm::errs() << OptionsParser.takeError();
    return 1;
  }
  clang::tooling::ClangTool Tool(OptionsParser->getCompilations(),
                                 OptionsParser->getSourcePathList());
  return Tool.run(clang::tooling::newFrontendActionFactory<
                    clangbook::recursivevisitor::FrontendAction>()
                    .get());
}
\end{cpp}

\begin{center}
图3.21: 递归访问者测试工具的主函数
\end{center}

在第23行只更改了自定义前端动作的命名空间名称。

前端动作和消费者的代码与图3.8和图3.9中的代码相同,唯一的区别是命名空间从declvisitor更改为recursivevisitor。程序中最有趣的部分是,访问者类的实现。

\begin{cpp}
#include "clang/AST/RecursiveASTVisitor.h"

namespace clangbook {
namespace recursivevisitor {
class Visitor : public clang::RecursiveASTVisitor<Visitor> {
public:
  bool VisitFunctionDecl(const clang::FunctionDecl *FD) {
    llvm::outs() << "Function: '" << FD->getName() << "'\n";
    return true;
  }
  bool VisitParmVarDecl(const clang::ParmVarDecl *PVD) {
    llvm::outs() << "\tParameter: '" << PVD->getName() << "'\n";
    return true;
  }
};
} // namespace recursivevisitor
} // namespace clangbook
\end{cpp}

\begin{center}
图3.22: 访问者类的实现
\end{center}

与'DeclVisitor'的代码(见图3.10)相比,有几个变化。首先,递归没有实现,只实现了对我们感兴趣的节点的回调。随之而来的是一个合理的问题:递归是如何控制的?答案在于另一个变化:回调现在返回一个布尔结果。false值表示递归应该停止,而true信号访问者继续遍历。

程序可以使用之前使用的相同的命令序列进行编译。见图3.11。

可以像这样运行程序,见图3.23:

\begin{shell}
$ ./recursivevisitor max.cpp -- -std=c++17
...
Function: 'max'
        Parameter: 'a'
        Parameter: 'b'
\end{shell}

\begin{center}
图3.23: 在测试文件上运行递归访问者实用程序的结果
\end{center}

它产生了与使用DeclVisitor实现获得的结果相同的输出。目前为止,考虑的AST遍历方法,并不是进行AST遍历的唯一方法。我们稍后考虑的大多数工具,将使用基于AST匹配器的不同方法。



































