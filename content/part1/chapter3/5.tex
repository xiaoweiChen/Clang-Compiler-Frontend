AST匹配器\footnote{LLVM Community. AST Matcher Reference. 2024. URL \url{https://clang.llvm.org/docs/LibASTMatchersReference.html}} 为定位特定的AST节点提供了另一种方法。在搜索不适当的模式,使用或识别需要修改的AST节点的检查器和重构工具。

我们将创建一个简单的程序来测试AST匹配。该程序将识别具有名称max的函数定义,使用从前面的示例中稍作修改的CMakeLists.txt文件,以包含支持AST匹配所需的库:

\begin{cmake}
cmake_minimum_required(VERSION 3.16)
project("matchvisitor")

if ( NOT DEFINED ENV{LLVM_HOME})
  message(FATAL_ERROR "$LLVM_HOME is not defined")
else()
  message(STATUS "$LLVM_HOME found: $ENV{LLVM_HOME}")
  set(LLVM_HOME $ENV{LLVM_HOME} CACHE PATH "Root of LLVM installation")
  set(LLVM_LIB ${LLVM_HOME}/lib)
  set(LLVM_DIR ${LLVM_LIB}/cmake/llvm)
  find_package(LLVM REQUIRED CONFIG)
  include_directories(${LLVM_INCLUDE_DIRS})
  link_directories(${LLVM_LIBRARY_DIRS})
  set(SOURCE_FILE MatchVisitor.cpp)
  add_executable(matchvisitor ${SOURCE_FILE})
  set_target_properties(matchvisitor PROPERTIES COMPILE_FLAGS "-fno-rtti")
  target_link_libraries(matchvisitor
    LLVMFrontendOpenMP
    LLVMSupport
    clangAST
    clangASTMatchers
    clangBasic
    clangFrontend
    clangSerialization
    clangTooling
  )
endif()
\end{cmake}

\begin{center}
图3.24: AST匹配器测试工具的CMakeLists.txt文件
\end{center}

添加了两个库:LLVMFrontendOpenMP和clangASTMatchers(见图3.24中的第18行和第21行)。工具的主函数看起来像这样:

\begin{cpp}
#include "clang/Tooling/CommonOptionsParser.h"
#include "clang/Tooling/Tooling.h"
#include "llvm/Support/CommandLine.h" // llvm::cl::extrahelp
#include "MatchCallback.hpp"

namespace {
llvm::cl::OptionCategory TestCategory("Test project");
llvm::cl::extrahelp
  CommonHelp(clang::tooling::CommonOptionsParser::HelpMessage);
} // namespace

int main(int argc, const char **argv) {
  llvm::Expected<clang::tooling::CommonOptionsParser> OptionsParser =
    clang::tooling::CommonOptionsParser::create(argc, argv, TestCategory);
  if (!OptionsParser) {
    llvm::errs() << OptionsParser.takeError();
    return 1;
  }
  clang::tooling::ClangTool Tool(OptionsParser->getCompilations(),
                                 OptionsParser->getSourcePathList());
  clangbook::matchvisitor::MatchCallback MC;
  clang::ast_matchers::MatchFinder Finder;
  Finder.addMatcher(clangbook::matchvisitor::M, &MC);
  return Tool.run(clang::tooling::newFrontendActionFactory(&Finder).get());
}
\end{cpp}

\begin{center}
图3.25: AST匹配器测试工具的主函数
\end{center}

第21-23行使用了MatchFinder类,并定义了一个自定义回调(通过第4行的头文件包含),该回调描述了我们打算匹配的特定AST节点。回调的实现如下:

\begin{cpp}
#include "clang/ASTMatchers/ASTMatchFinder.h"
#include "clang/ASTMatchers/ASTMatchers.h"

namespace clangbook {
namespace matchvisitor {
using namespace clang::ast_matchers;
static const char *MatchID = "match-id";
clang::ast_matchers::DeclarationMatcher M =
  functionDecl(decl().bind(MatchID), matchesName("max"));

class MatchCallback : public clang::ast_matchers::MatchFinder::MatchCallback {
public:
  virtual void
  run(const clang::ast_matchers::MatchFinder::MatchResult &Result) final {
    if (const auto *FD = Result.Nodes.getNodeAs<clang::FunctionDecl>(MatchID)) {
       const auto &SM = *Result.SourceManager;
       const auto &Loc = FD->getLocation();
       llvm::outs() << "Found 'max' function at " << SM.getFilename(Loc) << ":"
                    << SM.getSpellingLineNumber(Loc) << ":"
                    << SM.getSpellingColumnNumber(Loc) << "\n";
    }
   }
};


} // namespace matchvisitor
} // namespace clangbook
\end{cpp}

\begin{center}
图3.26: AST匹配器测试工具的匹配回调
\end{center}

代码中最重要的部分位于第7-9行。每个匹配器都由一个ID标识,在我们的示例中是'match-id'。匹配器本身在第8-9行定义:

\begin{cpp}
clang::ast_matchers::DeclarationMatcher M =
  functionDecl(decl().bind(MatchID), matchesName("max"));
\end{cpp}

这个匹配器寻找具有特定名称的函数声明,使用functionDecl(),正如在matchesName()中看到的那样。使用了一个专门的领域特定语言(DSL)来指定匹配器,DSL使用C++宏实现。还可以创建自己的匹配器,检查实现中展示。值得注意的是,递归AST访问者在匹配器实现中的AST遍历中起着核心作用。

程序可以使用之前使用的相同的命令序列进行编译。见图3.11。

使用图2.5中示例的一个修改版本,添加了一个函数:

\begin{cpp}
int max(int a, int b) {
  if (a > b) return a;
    return b;
}

int min(int a, int b) {
  if (a > b) return b;
    return a;
}
\end{cpp}

\begin{center}
图3.27: AST匹配器的测试程序minmax.cpp
\end{center}

运行测试工具时,将获得以下输出:

\begin{shell}
./matchvisitor minmax.cpp -- -std=c++17
...
Found the 'max' function at minmax.cpp:1:5
\end{shell}

\begin{center}
图3.28: 在测试文件上运行matchvisitor实用程序的结果
\end{center}

它只找到了与匹配器指定的名称匹配的一个函数声明。

匹配器的DSL通常在自定义Clang工具中使用,也可以作为独立工具使用。一个专门的程序名为clang-query,可以执行不同的匹配查询,用于在分析的C++代码中搜索特定的AST节点。让我们看看这个工具是如何工作的。










































