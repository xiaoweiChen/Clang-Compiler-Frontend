AST匹配器非常有用,有一个工具可以方便地检查各种匹配器并分析源代码的AST,这个工具就是clang-query。您可以使用以下命令来构建和安装这个工具:

\begin{shell}
$ ninja install-clang-query
\end{shell}

\begin{center}
图3.29:clang-query安装
\end{center}

您可以通过以下方式运行该工具:

\begin{shell}
$ <...>/llvm-project/install/bin/clang-query minmax.cpp
\end{shell}

\begin{center}
图3.30:在测试文件上运行clang-query
\end{center}

我们可以按以下方式使用匹配命令:

\begin{shell}
clang-query> match functionDecl(decl().bind("match-id"), matchesName("max"))
Match #1:
minmax.cpp:1:1: note: "match-id" binds here
int max(int a, int b) {
    ^~~~~~~~~~~~~~~~~~~~~~~
minmax.cpp:1:1: note: "root" binds here
int max(int a, int b) {
    ^~~~~~~~~~~~~~~~~~~~~~~
1  match.
clang-query>
\end{shell}

\begin{center}
图3.31:使用clang-query
\end{center}

图3.31展示了默认输出,称为'diag'。在众多可能的输出中,对我们来说最相关的是'dump'。当输出设置为'dump'时,clang-query将显示定位到的AST节点。例如,以下展示了如何匹配名为a的函数参数:

\begin{shell}
clang-query> set output dump
clang-query> match parmVarDecl(hasName("a"))
Match #1:
Binding for "root":
ParmVarDecl 0x6775e48 <minmax.cpp:1:9, col:13> col:13 used a 'int'
Match #2:
Binding for "root":
ParmVarDecl 0x6776218 <minmax.cpp:6:9, col:13> col:13 used a 'int'
2  matches.
clang-query>
\end{shell}

\begin{center}
图3.32:使用clang-query的dump输出
\end{center}

当您想要测试特定的匹配器或调查AST树的一部分时,这个工具非常有用。我们将使用这个工具来探索Clang如何处理编译错误。

















