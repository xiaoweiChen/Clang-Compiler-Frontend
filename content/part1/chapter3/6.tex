AST matchers are incredibly useful, and there's a utility that facilitates checking various matchers and analyzing the AST of your source code. This utility is known as clang-query tool. You can build and install this utility using the following command:

\begin{shell}
$ ninja install-clang-query
\end{shell}

\begin{center}
Figure 3.29: The clang-query installation
\end{center}

You can run the tool as follows:

\begin{shell}
$ <...>/llvm-project/install/bin/clang-query minmax.cpp
\end{shell}

\begin{center}
Figure 3.30: Running clang-query on a test file
\end{center}

We can use the match command as follows:

\begin{shell}
clang-query> match functionDecl(decl().bind("match-id"), matchesName("max"))
Match #1:
minmax.cpp:1:1: note: "match-id" binds here
int max(int a, int b) {
    ^~~~~~~~~~~~~~~~~~~~~~~
minmax.cpp:1:1: note: "root" binds here
int max(int a, int b) {
    ^~~~~~~~~~~~~~~~~~~~~~~
1  match.
clang-query>
\end{shell}

\begin{center}
Figure 3.31: Working with clang-query
\end{center}

Figure 3.31 demonstrates the default output, referred to as 'diag' . Among several potential outputs, the most relevant one for us is 'dump' . When the output is set to 'dump' , clang-query will display the located AST node. For example, the following demonstrates how to match a function parameter named a :

\begin{shell}
clang-query> set output dump
clang-query> match parmVarDecl(hasName("a"))
Match #1:
Binding for "root":
ParmVarDecl 0x6775e48 <minmax.cpp:1:9, col:13> col:13 used a 'int'
Match #2:
Binding for "root":
ParmVarDecl 0x6776218 <minmax.cpp:6:9, col:13> col:13 used a 'int'
2  matches.
clang-query>
\end{shell}

\begin{center}
Figure 3.32: Working with clang-query using dump output
\end{center}

This tool proves useful when you wish to test a particular matcher or investigate a portion of the AST tree. We will utilize this tool to explore how Clang handles compilation errors.

















