LLVM遵循特定的代码风格规则\footnote{LLVM Community. LLVM Coding Standards. 2023. URL \url{https://llvm.org/docs/CodingStandards.html}}。这些规则的主要目标是促进熟练的C++实践,特别关注性能。如前所述,LLVM使用C++17,并倾向于使用STL(标准模板库)中的数据结构和算法。另一方面,LLVM提供了许多优化版本的数据结构,这些数据结构模仿了STL中的对应结构。例如,llvm::SmallVector<>可以被视为std::vector<>的优化版本,特别是对于小尺寸的向量,这是编译器中使用的数据结构的常见特征。

在STL对象/算法及其对应的LLVM版本之间进行选择时,LLVM编码标准建议优先使用LLVM版本。

额外的规则涉及到性能限制的问题。例如,运行时类型信息(RTTI)和C++异常都是不允许的。然而,有些情况下RTTI可能会证明是有益的;因此,LLVM提供了替代方案,如llvm::isa<>以及其他类似的模板辅助函数。有关更多信息,请参见第4.3.1节,RTTI替换和强制转换运算符。LLVM经常使用C风格的断言来替代C++异常。

有时,断言并不足够信息丰富。LLVM建议向它们添加文本消息以简化调试。以下是Clang代码中的一个典型示例:

\begin{cpp}
static bool unionHasUniqueObjectRepresentations(const ASTContext &Context,
  const RecordDecl *RD,
  bool CheckIfTriviallyCopyable) {

    assert(RD->isUnion() && "Must be union type");
    CharUnits UnionSize = Context.getTypeSizeInChars(RD->getTypeForDecl());
\end{cpp}

\begin{center}
图4.1:在clang/lib/AST/ASTContext.cpp中使用assert()
\end{center}

在代码中,我们检查第二个参数(RD)是否为联合体,并在不是的情况下使用相应的消息引发断言。

除了性能考虑之外,LLVM还引入了一些额外的要求。其中之一涉及到注释。代码注释非常重要。此外,LLVM和Clang都有从代码生成的全面文档。它们使用Doxygen(\url{https://www.doxygen.nl/})来实现这一点。这个工具是C/C++程序中注释的事实标准,您可能之前已经遇到过。

Clang和LLVM不是单一的大块代码;相反,它们被实现为一组库。这种设计在代码和功能重用方面提供了优势,我们将在第8章"IDE支持和Clangd"中探讨这一点。这些库也是LLVM代码风格执行的优秀示例。让我们详细检查这些库。



































































































