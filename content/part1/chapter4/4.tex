Clang是一个编译器前端,其最重要的操作与诊断相关。诊断需要精确的源代码位置信息。让我们探索Clang为这些操作提供的基本类。

\mySubsubsection{4.4.1.}{SourceManager 和 SourceLocation}

Clang作为一个编译器,操作文本文件(程序),而定位程序中的特定位置是最常请求的操作。典型的Clang错误报告,Clang为该程序生成的错误消息如下:

\begin{shell}
$ <...>/llvm-project/install/bin/clang -fsyntax-only maxerr.cpp
maxerr.cpp:3:12: error: use of undeclared identifier 'ab'
  return ab;
         ^
1  error generated.
\end{shell}

\begin{center}
图4.12:maxerr.cpp中的错误报告
\end{center}

如图4.12所示,显示消息所需的信息包括:

\begin{itemize}
\item
文件名: 在我们的例子中,它是maxerr.cpp

\item
文件行:在我们的例子中,它是3

\item
文件列:在我们的例子中,它是12
\end{itemize}

存储这些信息的结构体应该尽可能紧凑,因为编译器经常使用它。Clang将这些信息存储在clang::SourceLocation对象中。

这个对象经常使用,应该尽可能小且快速复制。可以使用lldb检查对象的大小。如果使用调试器运行Clang,可以确定其大小:

\begin{shell}
$ lldb <...>/llvm-project/install/clang
...
(lldb) p sizeof(clang::SourceLocation)
(unsigned long) 4
(lldb)
\end{shell}

\begin{center}
图4.13:在调试器下确定clang::SourceLocation的大小
\end{center}

信息是通过一个无符号长整数编码,这是如何做到的呢?这个数字仅仅作为文本文件中位置的标识符。另外,需要一个额外的类来正确提取和表示这些信息,这个类是clang::SourceManager,SourceManager对象包含有关特定位置的详细信息。Clang中,管理源位置可能具有挑战性,因为存在宏、包含和其他预处理指令。有几种方式可以解释给定的源位置:

\begin{itemize}
\item
拼写位置:指的是某个东西在源中实际拼写出来的位置。如果有一个指向宏体内部的源位置,拼写位置将给你源代码中宏内容定义的位置。

\item
扩展位置:指的是宏在哪里被扩展。如果有一个指向宏体内部的源位置,扩展位置将给你源代码中使用宏(扩展)的位置。
\end{itemize}

看一个具体的例子:

\begin{cpp}
#define BAR void bar()
int foo(int x);
BAR;
\end{cpp}

\begin{center}
图4.14:测试不同类型的源位置的示例程序:functions.hpp
\end{center}

图4.14中,第1行定义了两个函数:int foo()在第2行和void bar()在第3行。对于第一个函数,拼写位置和扩展位置都指向第2行。对于第二个函数,拼写位置在第1行,而扩展位置在第3行。

用一个测试Clang工具来检查这个例子。将使用第3.4节的测试项目,Recursive AST visitor,并在此处替换一些代码部分。首先,必须将clang::ASTContext传递给我们的Visitor实现,因为clang::ASTContext提供对clang::SourceManager的访问。替换图3.8中的第11行,并如下传递ASTContext:

\begin{cpp}
CreateASTConsumer(clang::CompilerInstance &CI, llvm::StringRef File) {
  return std::make_unique<Consumer>(&CI.getASTContext());
\end{cpp}

Consumer类(见图3.9)将接受参数并将其作为Visitor的参数使用:

\begin{cpp}
Consumer(clang::ASTContext *Context)
  : V(std::make_unique<Visitor>(Context)) {}
\end{cpp}

主要的修改是在Visitor类中,几乎重写。首先,将clang::ASTContext传递给类构造函数:

\begin{cpp}
class Visitor : public clang::RecursiveASTVisitor<Visitor> {
public:
  explicit Visitor(clang::ASTContext *C) : Context(C) {}
\end{cpp}

\begin{center}
图4.15:Visitor类实现:构造函数
\end{center}

AST Context类作为私有成员存储:

\begin{cpp}
private:
  clang::ASTContext *Context;
\end{cpp}

\begin{center}
图4.16:Visitor类实现:private部分
\end{center}

主要处理逻辑在Visitor::VisitFunctionDecl方法中:

\begin{cpp}
bool VisitFunctionDecl(const clang::FunctionDecl *FD) {
  clang::SourceManager &SM = Context->getSourceManager();
  clang::SourceLocation Loc = FD->getLocation();
  clang::SourceLocation ExpLoc = SM.getExpansionLoc(Loc);
  clang::SourceLocation SpellLoc = SM.getSpellingLoc(Loc);
  llvm::StringRef ExpFileName = SM.getFilename(ExpLoc);
  llvm::StringRef SpellFileName = SM.getFilename(SpellLoc);
  unsigned SpellLine = SM.getSpellingLineNumber(SpellLoc);
  unsigned ExpLine = SM.getExpansionLineNumber(ExpLoc);
  llvm::outs() << "Spelling : " << FD->getName() << " at " << SpellFileName
               << ":" << SpellLine << "\n";
  llvm::outs() << "Expansion : " << FD->getName() << " at " << ExpFileName
               << ":" << ExpLine << "\n";
  return true;
}
\end{cpp}

\begin{center}
图4.17:Visitor类实现:VisitFunctionDecl方法
\end{center}

如果编译并运行图4.14中的测试文件,输出为:

\begin{shell}
Spelling : foo at functions.hpp:2
Expansion : foo at functions.hpp:2
Spelling : bar at functions.hpp:1
Expansion : bar at functions.hpp:3
\end{shell}

\begin{center}
图4.18:recursivevisitor可执行文件在functions.hpp测试文件上的输出
\end{center}

clang::SourceLocation和clang::SourceManager是非常强大的类。与诸如clang::SourceRange(两个源位置的配对,用于指定源范围的开始和结束)等其他类结合使用,为Clang中使用的诊断提供了坚实的基础。

\mySubsubsection{4.4.2.}{诊断支持}

Clang的诊断子系统负责生成和报告警告、错误和其他消息\footnote{LLVM Community. "Clang" CFE Internals Manual. 2023. URL \url{https://clang.llvm.org/docs/InternalsManual.html}}。涉及的主要类包括:

\begin{itemize}
\item
DiagnosticsEngine: 管理诊断ID和选项

\item
DiagnosticConsumer: 诊断消费者的抽象基类

\item
DiagnosticIDs: 处理诊断标志和内部ID之间的映射

\item
DiagnosticInfo: 单个诊断
\end{itemize}

以下是一个简单的例子,展示了如何在Clang中发出警告:

\begin{cpp}
// Emit a warning
DiagnosticsEngine.Report(DiagnosticsEngine.getCustomDiagID(
  clang::DiagnosticsEngine::Warning, "This is a custom warning."));
\end{cpp}

\begin{center}
图4.19:使用clang::DiagnosticsEngine发出警告
\end{center}

我们将使用一个简单的DiagnosticConsumer,即clang::TextDiagnosticPrinter,格式化和打印处理后的诊断消息。

示例中主函数的完整代码如图4.20所示:

\begin{cpp}
int main() {
  llvm::IntrusiveRefCntPtr<clang::DiagnosticOptions> DiagnosticOptions =
    new clang::DiagnosticOptions();
  clang::TextDiagnosticPrinter TextDiagnosticPrinter(
    llvm::errs(), DiagnosticOptions.get(), false);

  llvm::IntrusiveRefCntPtr<clang::DiagnosticIDs> DiagIDs =
    new clang::DiagnosticIDs();
  clang::DiagnosticsEngine DiagnosticsEngine(DiagIDs, DiagnosticOptions,
                                             &TextDiagnosticPrinter, false);

  // Emit a warning
  DiagnosticsEngine.Report(DiagnosticsEngine.getCustomDiagID(
    clang::DiagnosticsEngine::Warning, "This is a custom warning."));

  return 0;
}
\end{cpp}

\begin{center}
图4.20:Clang诊断示例
\end{center}

这段代码将产生以下输出:

\begin{shell}
warning: This is a custom warning.
\end{shell}

\begin{center}
图4.21:打印的诊断信息
\end{center}

这个例子中,首先使用TextDiagnosticPrinter作为其DiagnosticConsumer来设置DiagnosticsEngine。然后,使用DiagnosticsEngine的Report方法发出一个自定义警告。我们将在创建第4.6节中的Clang插件的测试项目时添加一个更现实的例子。

























































