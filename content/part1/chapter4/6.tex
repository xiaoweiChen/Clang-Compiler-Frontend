测试项目的目标是创建一个Clang插件,用于估算类的复杂性。具体来说,如果一个类的方法数量超过某个阈值,那么这个类就被认为是复杂的。我们将利用到目前为止所获得的所有知识来完成这个项目。这包括使用递归访问者和Clang诊断。此外,我们还将为我们的项目创建一个LIT测试。开发插件将需要一个独特的LLVM构建配置,这将是我们首先要做的事情。


\mySubsubsection{4.6.1.}{环境设置}

插件将作为一个共享对象创建,我们的LLVM安装应该支持共享库(见第1.3.1节,使用CMake配置):

\begin{shell}
cmake -G Ninja -DCMAKE_BUILD_TYPE=Debug -DCMAKE_INSTALL_PREFIX=../install -DLLVM_TARGETS_TO_BUILD="X86" -DLLVM_ENABLE_PROJECTS="clang" -DLLVM_USE_SPLIT_DWARF=ON -DBUILD_SHARED_LIBS=ON ../llvm
\end{shell}

\begin{center}
图4.31:用于Clang插件项目的CMake配置
\end{center}

如图所示,我们使用了第1.4节中的构建配置,如图1.12所示。在配置中,我们为安装工件设置了一个文件夹 ../install,将构建目标限制为X86平台,并只启用clang项目。此外,我们启用了调试符号的大小优化,并使用共享库而不是静态链接。

接下来,我们需要构建和安装clang。这可以通过以下命令完成:

\begin{shell}
$ ninja install
\end{shell}

一旦我们完成了clang的构建和安装,我们就可以继续编写我们的项目CMakeLists.txt文件。

\mySubsubsection{4.6.2.}{插件的CMake构建配置}

我们将使用图3.20作为我们的插件构建配置的基础。我们将更改项目名称为classchecker,而ClassComplexityChecker.cpp将作为我们的主要源文件。文件的主要部分如图4.32所示。如图所示,我们将构建一个共享库(第18-20行),而不是之前的测试项目中的可执行文件。另一个修改是在第12行,我们设置了一个配置参数来定位LLVM构建文件夹。这个参数是必要的,以便找到LIT可执行文件,正如之前在第4.5.2节,LLVM测试框架中提到的,LIT可执行文件不包括在标准安装过程中。为了支持LIT测试的调用,我们还需要进行一些额外的修改,但我们会稍后在图4.44中讨论这些细节。



\begin{cmake}
message(STATUS "$LLVM_HOME found: $ENV{LLVM_HOME}")
set(LLVM_HOME $ENV{LLVM_HOME} CACHE PATH "Root of LLVM installation")
set(LLVM_LIB ${LLVM_HOME}/lib)
set(LLVM_DIR ${LLVM_LIB}/cmake/llvm)
set(LLVM_BUILD $ENV{LLVM_BUILD} CACHE PATH "Root of LLVM build")
find_package(LLVM REQUIRED CONFIG)
include_directories(${LLVM_INCLUDE_DIRS})
link_directories(${LLVM_LIBRARY_DIRS})

# Add the plugin's shared library target
add_library(classchecker MODULE
  ClassChecker.cpp
)
set_target_properties(classchecker PROPERTIES COMPILE_FLAGS "-fno-rtti")
target_link_libraries(classchecker
  LLVMSupport
  clangAST
  clangBasic
  clangFrontend
  clangTooling
)
\end{cmake}

\begin{center}
图4.32:class complexity插件的CMakeLists.txt文件
\end{center}

完成构建配置后,我们可以开始编写插件的主要代码。我们将创建的第一个组件是一个递归访问者类,名为ClassVisitor。

\mySubsubsection{4.6.3.}{递归访问者类}

我们的访问者类位于ClassVisitor.hpp文件中(见图4.33)。这是一个递归访问者,处理clang::CXXRecordDecl,这是C++类声明的AST节点。我们在第13-16行计算方法的数量,并在超过阈值时在第19-25行发出诊断。

\begin{cpp}
#include "clang/AST/ASTContext.h"
#include "clang/AST/RecursiveASTVisitor.h"

namespace clangbook {
namespace classchecker {
class ClassVisitor : public clang::RecursiveASTVisitor<ClassVisitor> {
public:
  explicit ClassVisitor(clang::ASTContext *C, int T)
    : Context(C), Threshold(T) {}

  bool VisitCXXRecordDecl(clang::CXXRecordDecl *Declaration) {
    if (Declaration->isThisDeclarationADefinition()) {
      int MethodCount = 0;
      for (const auto *M : Declaration->methods()) {
        MethodCount++;
      }
      if (MethodCount > Threshold) {
        clang::DiagnosticsEngine &D = Context->getDiagnostics();
        unsigned DiagID =
          D.getCustomDiagID(clang::DiagnosticsEngine::Warning,
                            "class %0 is too complex: method count = %1");
        clang::DiagnosticBuilder DiagBuilder =
               D.Report(Declaration->getLocation(), DiagID);
        DiagBuilder << Declaration->getName() << MethodCount;
      }
    }
    return true;
  }

private:
  clang::ASTContext *Context;
  int Threshold;
};
} // namespace classchecker
} // namespace clangbook
\end{cpp}


\begin{center}
图4.33:ClassVisitor.hpp的源代码
\end{center}

值得注意的诊断调用。诊断消息在行20-22中构建。我们的诊断消息接受两个参数:类名和类的函数数量。这些参数在第22行以'\%1'和'\%2'占位符编码。这些参数的实际值在第25行传递,在那里使用DiagBuild对象构建诊断消息。这个对象是clang::DiagnosticBuilder类的实例,它实现了资源获取即初始化(RAII)模式。当对象超出作用域时,它会自动调用其析构函数,并在此时发出实际的诊断信息。

\begin{myNotic}{重要提示}
在C++中,RAII原则是一种常见的习惯用法,用于通过将资源与对象的生存期联系起来来管理资源生命周期。当对象超出作用域时,其析构函数会自动被调用,这为释放对象持有的资源提供了机会。
\end{myNotic}

ClassVisitor是在AST消费者类中创建的,这是我们的下一个主题。

\mySubsubsection{4.6.4.}{插件AST消费者类}

AST消费者类在ClassConsumer.hpp中实现,并在我们的AST访问者测试项目中代表标准AST消费者(参见图3.9)。代码在图4.35中展示。

\begin{cpp}
namespace clangbook {
namespace classchecker {
class ClassConsumer : public clang::ASTConsumer {
public:
  explicit ClassConsumer(clang::ASTContext *Context, int Threshold)
    : Visitor(Context, Threshold) {}

  virtual void HandleTranslationUnit(clang::ASTContext &Context) {
    Visitor.TraverseDecl(Context.getTranslationUnitDecl());
  }

private:
  ClassVisitor Visitor;
};
} // namespace classchecker
} // namespace clangbook
\end{cpp}

\begin{center}
图4.34:ClassConsumer.hpp的源代码
\end{center}

代码在第10行初始化Visitor,并在第13行使用Visitor类遍历声明,从最顶层的声明(翻译单元声明)开始。消费者必须从特殊的AST动作类创建,这是我们接下来要讨论的。

\mySubsubsection{4.6.5.}{插件AST动作类}

插件AST动作的代码如图4.35所示。我们可以观察到几个重要的部分:

\begin{itemize}
\item
第7行:我们从clang::PluginASTAction派生我们的ClassAction。

\item
第10-13行:我们实例化ClassConsumer并利用MethodCountThreshold,这是从可选插件参数派生的。

\item
第15-25行:我们处理插件的可选阈值参数
\end{itemize}

\begin{cpp}
namespace clangbook {
namespace classchecker {
class ClassAction : public clang::PluginASTAction {
protected:
  std::unique_ptr<clang::ASTConsumer>
  CreateASTConsumer(clang::CompilerInstance &CI, llvm::StringRef) {
    return std::make_unique<ClassConsumer>(&CI.getASTContext(),
                                           MethodCountThreshold);
  }

  bool ParseArgs(const clang::CompilerInstance &CI,
    const std::vector<std::string> &args) {
      for (const auto &arg : args) {
        if (arg.substr(0, 9) == "threshold") {
          auto valueStr = arg.substr(10); // Get the substring after "threshold="
          MethodCountThreshold = std::stoi(valueStr);
          return true;
        }
      }
    return true;
  }
  ActionType getActionType() { return AddAfterMainAction; }

private:
  int MethodCountThreshold = 5; // default value
};
} // namespace classchecker
} // namespace clangbook
\end{cpp}

\begin{center}
图4.35:ClassAction.hpp的源代码
\end{center}

我们几乎完成了,并准备初始化我们的插件。

\mySubsubsection{4.6.6.}{插件代码}

我们的插件注册在ClassChecker.cpp文件中,如图4.36所示。

\begin{cpp}
#include "clang/Frontend/FrontendPluginRegistry.h"

#include "ClassAction.hpp"

static clang::FrontendPluginRegistry::Add<clangbook::classchecker::ClassAction>
X("classchecker", "Checks the complexity of C++ classes");
\end{cpp}


\begin{center}
图4.36:ClassChecker.cpp的源代码
\end{center}

如我们所观察到的,大多数初始化都隐藏在辅助类中,我们只需要将我们的实现传递给lang::FrontendPluginRegistry::Add。

现在我们准备构建和测试我们的Clang插件。

\mySubsubsection{4.6.7.}{构建和运行插件代码}

我们需要指定LLVM项目的安装文件夹的路径。其余的步骤是我们之前使用过的标准步骤,见图3.11:

\begin{shell}
export LLVM_HOME=<...>/llvm-project/install
mkdir build
cd build
cmake -G Ninja -DCMAKE_BUILD_TYPE=Debug ..
ninja classchecker
\end{shell}

\begin{center}
图4.37:为Clang插件配置和构建命令
\end{center}

构建工件将位于build文件夹中。然后,我们可以像以下这样在测试文件上运行我们的插件,其中是要编译的文件路径:

\begin{shell}
$ <...>/llvm-project/install/bin/clang -fsyntax-only \
                 -fplugin=./build/libclasschecker.so \
                 <filepath>
\end{shell}

\begin{center}
图4.38:如何在测试文件上运行Clang插件
\end{center}

例如,如果我们使用一个定义了具有三个方法的类的测试文件test.cpp(见图4.39),我们将不会收到任何警告。

\begin{cpp}
class Simple {
public:
  void func1() {}
  void func2() {}
  void func3() {}
};
\end{cpp}

\begin{center}
图4.39:Clang插件的测试:test.cpp
\end{center}

然而,如果我们指定一个更小的阈值,我们将为该文件收到警告:

\begin{shell}
$ <...>/llvm-project/install/bin/clang -fsyntax-only \
                  -fplugin-arg-classchecker-threshold=2 \
                  -fplugin=./build/libclasschecker.so \
                  test.cpp
test.cpp:1:7: warning: class Simple is too complex: method count = 3
  1 | class Simple {
    |       ^
1  warning generated.
\end{shell}


\begin{center}
图4.40:Clang插件在test.cpp上运行
\end{center}

现在是时候为我们的插件创建一个LIT测试了。

\mySubsubsection{4.6.8.}{LIT测试用于Clang插件}

我们将从一个项目组织描述开始。我们将采用Clang源代码中常用的模式,并将我们的测试放在test文件夹中。这个文件夹将包含以下文件:

\begin{itemize}
\item
lit.site.cfg.py.in : 这是主要的配置文件,一个CMake配置文件。它将标记为'@…@'的模式替换为在CMake配置中定义的相应值。此外,这个文件还加载了lit.cfg.py。

\item
lit.cfg.py : 这是LIT测试的主要配置文件。

\item
simple\_test.cpp : 这是我们的LIT测试文件。
\end{itemize}

基本工作流程如下:CMake将lit.site.cfg.py.in作为模板,并在build/test文件夹中生成相应的lit.site.cfg.py。这个文件然后被LIT测试用作执行测试的种子。

\mySamllsection{LIT配置文件}

有两个LIT测试配置文件。第一个在图4.41中显示。

\begin{python}
config.ClassComplexityChecker_obj_root = "@CMAKE_CURRENT_BINARY_DIR@"
config.ClassComplexityChecker_src_root = "@CMAKE_CURRENT_SOURCE_DIR@"
config.ClangBinary = "@LLVM_HOME@/bin/clang"
config.FileCheck = "@FILECHECK_COMMAND@"

lit_config.load_config(
  config, os.path.join(config.ClassComplexityChecker_src_root, "test/lit.cfg.py"))
\end{python}

\begin{center}
图4.41:lit.site.cfg.py.in文件
\end{center}

这个文件是一个CMake模板,将被转换成Python脚本。最关键的部分是第6-7行,其中主要LIT配置被加载。它来自主源树,并且不会被复制到build文件夹。

随后的配置在图4.42中显示。这是一个Python脚本,包含LIT测试的主要配置。

\begin{python}
# lit.cfg.py
import lit.formats

config.name = 'classchecker'
config.test_format = lit.formats.ShTest(True)
config.suffixes = ['.cpp']
config.test_source_root = os.path.dirname(__file__)

config.substitutions.append(('%clang-binary', config.ClangBinary))
config.substitutions.append(('%path-to-plugin', os.path.join(config.ClassComplexityChecker_obj_root, 'libclasschecker.so')))
config.substitutions.append(('%file-check-binary', config.FileCheck))
\end{python}

\begin{center}
图4.42:lit.cfg.py文件
\end{center}

第4-7行定义了基本配置;例如,第6行决定了哪些文件应该用于测试。测试文件夹中的所有以'.cpp'扩展名的文件都将被用作LIT测试。

第9-11行详细说明了LIT测试中将使用的替换。这些包括clang二进制文件的路径(第9行)、包含插件的共享库的路径(第10行)以及FileCheck实用程序的路径(第11行)。

我们只定义了一个基本的LIT测试,simple\_test.cpp,如图4.43所示。

\begin{cpp}
// RUN: %clang-binary -fplugin=%path-to-plugin -fsyntax-only %s 2>&1 | %file-check-binary %s

class Simple {
public:
  void func1() {}
  void func2() {}
};

// CHECK: :[[@LINE+1]]:{{[0-9]+}}: warning: class Complex is too complex: method count = 6
class Complex {
public:
  void func1() {}
  void func2() {}
  void func3() {}
  void func4() {}
  void func5() {}
  void func6() {}
};
\end{cpp}

\begin{center}
图 4.43: simple\_test.cpp 文件
\end{center}

在这个文件中,我们可以观察到第1行中使用的替换。路径指向clang二进制文件、插件共享库和FileCheck实用程序。在第9行中使用了特殊模式,这些模式被工具识别。

最后一块拼图是CMake配置。这将设置lit.site.cfg.py.in中所需的变量,以便进行替换,并且定义了一个自定义目标来运行LIT测试。

\mySamllsection{LIT测试的CMake配置}

为了支持LIT测试,CMakeLists.txt文件需要进行一些调整。必要的变化在图4.44中显示。

\begin{cmake}
find_program(LIT_COMMAND llvm-lit PATH ${LLVM_BUILD}/bin)
find_program(FILECHECK_COMMAND FileCheck ${LLVM_BUILD}/bin)
if(LIT_COMMAND AND FILECHECK_COMMAND)
  message(STATUS "$LIT_COMMAND found: ${LIT_COMMAND}")
  message(STATUS "$FILECHECK_COMMAND found: ${FILECHECK_COMMAND}")

  # Point to our custom lit.cfg.py
  set(LIT_CONFIG_FILE "${CMAKE_CURRENT_SOURCE_DIR}/test/lit.cfg.py")

  # Configure lit.site.cfg.py using current settings
  configure_file("${CMAKE_CURRENT_SOURCE_DIR}/test/lit.site.cfg.py.in"
                 "${CMAKE_CURRENT_BINARY_DIR}/test/lit.site.cfg.py"
                 @ONLY)

  # Add a custom target to run tests with lit
  add_custom_target(check-classchecker
                    COMMAND ${LIT_COMMAND} -v ${CMAKE_CURRENT_BINARY_DIR}/test
                    COMMENT "Running lit tests for classchecker clang plugin"
                    USES_TERMINAL)
else()
  message(FATAL_ERROR "It was not possible to find the LIT executables at ${LLVM_BUILD}/bin")
endif()
\end{cmake}

\begin{center}
图4.44:在CMakeLists.txt中的LIT测试配置
\end{center}

在第31行和第32行,我们搜索必要的实用程序,llvm-lit和FileCheck。值得注意的是,它们依赖于\$LLVM\_BUILD环境变量,我们在配置文件(见图4.32的第12行)中也验证了这个变量。第41-43行是生成lit.site.cfg.py的关键步骤,该文件来自提供的模板文件lit.site.cfg.py.in。最后,在第46-49行,我们建立了一个自定义目标来执行LIT测试。

现在我们准备开始LIT测试。

\mySamllsection{运行LIT测试}

要启动LIT测试,我们必须设置一个环境变量,该变量指向构建文件夹,编译项目,然后执行自定义目标check-classchecker。以下是执行这些操作的方法:

\begin{shell}
xport LLVM_BUILD=<...>/llvm-project/build
export LLVM_HOME=<...>/llvm-project/install
rm -rf build; mkdir build; cd build
cmake -G Ninja -DCMAKE_BUILD_TYPE=Debug ..
ninja classchecker
ninja check-classchecker
\end{shell}

\begin{center}
图4.45:Clang插件的配置、构建和检查命令
\end{center}

执行这些命令后,您可能会看到以下输出:

\begin{shell}
...
[2/2] Linking CXX shared module libclasschecker.so
[0/1] Running lit tests for classchecker clang plugin
-- Testing: 1 tests, 1 workers --
PASS: classchecker :: simple_test.cpp (1 of 1)

Testing Time: 0.12s
Passed: 1
\end{shell}

\begin{center}
图4.46:LIT测试执行
\end{center}

通过这个,我们完成了我们的第一个综合项目,它包含了一个实用的Clang插件,可以通过补充插件参数进行定制。此外,它还包括相应的测试,可以执行以验证其功能。








