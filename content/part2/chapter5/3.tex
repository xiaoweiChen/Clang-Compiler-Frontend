


Clang-Tidy是建立在Clang之上的。在其核心部分,Clang-Tidy利用Clang将源代码解析和分析成抽象语法树(AST)的能力。Clang-Tidy中的每个检查本质上都涉及定义与这个AST匹配的模式或条件。当找到匹配项时,可以提出诊断,在许多情况下,还可以建议自动修复。该工具基于针对特定问题或编码风格的单独“检查”进行操作。检查以插件的形式实现,使得Clang-Tidy具有可扩展性。ASTMatchers库通过提供一种特定领域的语言来查询AST,从而简化了这些检查的编写。这确保了检查既简洁又有表现力。Clang-Tidy还支持使用编译数据库来分析代码库,该数据库提供了编译标志等上下文信息(更多信息请参见第9章,附录1:编译数据库)。这种与Clang内部机制的全面集成,使得Clang-Tidy成为一个具有精确代码转换能力的强大静态分析工具。


\mySubsubsection{5.3.1.}{内部组织}

Clang-Tidy在Clang代码库中的内部组织可能因为其与Clang库的深度集成而复杂,但从高层次来看,组织可以分解如下:

\begin{enumerate}
\item
源代码和头文件:clang-tidy的主要源代码和头文件位于clang-tools-extra仓库中的clang-tidy目录。

\item
主驱动程序:位于工具子文件夹中的ClangTidyMain.cpp文件,是Clang-Tidy工具的主驱动程序。

\item
核心基础设施:如ClangTidy.cpp、ClangTidy.h等文件,管理核心功能和选项。

\item
检查:检查按类别(例如bugprone或modernize)组织成子目录。

\item
实用工具:utils目录包含实用类和函数。

\item
AST 匹配器:我们在第3.5节中探索的ASTMatchers库对于查询AST至关重要。

\item
Clang诊断:Clang-Tidy积极使用Clang的诊断子系统来打印诊断消息和建议修复。

\item
测试:测试位于测试目录中,并使用LLVM的LIT框架。值得注意的是,测试文件夹在clang-tools-extra文件夹内与其他项目共享。

\item
文档:docs目录包含Clang-Tidy的文档。与测试一样,文档是clang-tools-extra文件夹中其他项目的一部分。
\end{enumerate}

这些关系在以下图中进行了示意性说明:

\myGraphic{0.7}{content/part2/chapter5/images/1.png}{图5.11:Clang-Tidy的内部组织}

现在我们已经了解了Clang-Tidy的内部设计和与其他Clang/LLVM部分的关联,是时候探索与Clang-Tidy二进制文件外部相关的组件:它的配置和其他利用Clang-Tidy提供的功能的工具。

\mySubsubsection{5.3.2.}{配置和集成}

Clang-Tidy二进制文件可以与其他组件交互,如图5.12所示。

\myGraphic{0.5}{content/part2/chapter5/images/2.png}{图5.12:Clang-Tidy外部组件:配置和集成}

Clang-Tidy可以无缝集成到各种集成开发环境(IDE)中,例如Visual Studio Code、CLion和Eclipse,以在编码过程中提供实时反馈。我们将在第8.5.2节中探索这一可能性。

它还可以集成到构建系统,如CMake和Bazel,以便在构建过程中运行检查。持续集成(CI)平台,如Jenkins和GitHub Actions,经常使用Clang-Tidy来确保拉请求上的代码质量。代码审查平台,如Phabricator,利用Clang-Tidy进行自动化审查。此外,自定义脚本和静态分析平台可以利用Clang-Tidy的功能进行定制的工作流程和组合分析。

图5.12中另一个重要的部分是Clang-Tidy的配置。让我们详细探讨它。

\mySamllsection{Clang-Tidy 配置}

Clang-Tidy使用配置文件来指定要运行哪些检查以及为这些检查设置选项。这个配置是通过.clang-tidy文件完成的。

.clang-tidy文件采用YAML格式编写。它通常包含两个主要键:Checks和CheckOptions。

我们将从Checks键开始,它允许我们指定要启用或禁用的检查:

\begin{itemize}
\item
使用 - 来禁用一个检查

\item
使用 * 作为通配符来匹配多个检查

\item
检查之间用逗号分隔
\end{itemize}

这里有一个例子:

\begin{shell}
Checks: '-*,modernize-*'
\end{shell}

\begin{center}
图5.13:.clang-tidy配置文件中的Checks键
\end{center}

下一个键是CheckOptions。这个键允许我们为特定检查设置选项,每个选项以键值对的形式指定。这里提供了一个例子:

\begin{shell}
CheckOptions:
- key: readability-identifier-naming.NamespaceCase
  value: CamelCase
- key: readability-identifier-naming.ClassCase
  value: CamelCase
\end{shell}

\begin{center}
图5.14:.clang-tidy配置文件中的CheckOptions键
\end{center}

当Clang-Tidy运行时,它会在正在处理的文件及其父目录中搜索.clang-tidy文件。当找到文件时,搜索停止。

现在我们已经了解了Clang-Tidy的内部设计,是时候使用我们从本书的这一章和之前的章节中收集的信息来创建我们第一个自定义的Clang-Tidy检查了。




























