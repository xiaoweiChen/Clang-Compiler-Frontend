静态分析是软件开发中的一个关键技术,它涉及在不实际运行程序的情况下检查代码。这种方法专注于分析源代码或其编译版本,以检测各种问题,如错误、漏洞和违反编码标准的情况。与动态分析不同,后者需要在程序执行时进行检查,静态分析允许在非运行时环境中检查代码。

更广泛地说,静态分析旨在根据程序的意义检查特定属性;也就是说,它可以被认为是语义分析的一部分(参见图 2.6,解析器)。例如,如果 $\mathcal{C}$ 是所有 C/C++ 程序的集合,而 $\mathcal{P}$ 是这类程序的一种属性,那么静态分析的目标就是针对特定程序 $\mathcal{P} \in \mathcal{C}$,判断 $\mathcal{P}(P)$ 是否为真或假。

我们在前一章中创建的 Clang-Tidy 检查(参见第 5.4 节,自定义 Clang-Tidy 检查)就是一个这样的属性的例子。实际上,它接收带有类定义的 C++ 代码,并根据类的方法数量决定该类是否复杂。

值得注意的是,并非所有程序属性都可以被检查。最明显的例子就是著名的停机问题。 \footnote{Alan M. Turing. On Computable Numbers, with an Application to the Entscheidungsproblem. Proceedings of the London Mathematical Society, s2-42(1):230–265, 1937. doi: 10.1112/plms/s2-42.1.230}。

\begin{myNotic}{重要说明}
停机问题可以这样表述:给定一个程序 P 和输入 I,确定当使用 I 执行 P时,P是否会停止运行或无限期地继续运行。
\end{myNotic}

正式地说,问题是要决定,对于给定的程序P和输入I,P(I)的计算是否最终停止(停止)或永远不会终止(无限循环)。

艾伦·图灵证明了,对于所有可能的程序-输入对,没有通用的算法方法可以解决这个问题的。这意味着没有单一的算法可以正确确定每一对(P,I)在运行I时P是否停止。

尽管并非所有程序的属性都可以证明,但在某些情况下可以做到这一点。有一些这样的案例使得静态分析成为一个实用的工具。因此,在这些情况下,我们可以使用这些工具系统地扫描代码,以确定代码的属性。这些工具擅长识别从简单的语法错误到更复杂的潜在错误。静态分析的一个关键优点是它能够尽早地在开发周期中捕捉到问题。这种早期检测不仅高效,而且节省资源,因为它帮助我们在软件运行或部署之前识别和修复问题。

静态分析在确保软件质量和合规性方面发挥着重要作用。它检查代码是否符合规定的编码标准和指南,这在大型项目或具有严格监管要求的行业中尤为重要。此外,它非常有效于揭示常见的安全漏洞,如缓冲区溢出、SQL注入缺陷和跨站脚本漏洞。

此外,静态分析通过指出冗余、不必要的复杂性和改进的机会,有助于代码重构和优化。将这类工具集成到开发流程中,包括持续集成管道,是一种常见的做法。这种集成允许对每次新的提交或构建进行持续的代码分析,确保持续的质量保证。

我们在上一章中创建的Clang-Tidy检查可以被视为一个静态分析程序的例子。在本章中,我们将考虑涉及数据结构如CFGs的更高级主题,我们将在下一节看到。
