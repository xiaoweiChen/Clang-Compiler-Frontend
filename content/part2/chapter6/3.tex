控制流图(CFG)是编译器设计和静态程序分析中的一个基本数据结构,它表示程序在执行过程中可能遍历的所有路径。

CFG包含以下关键组件:

\begin{itemize}
\item
节点:对应于基本块,是一系列直线的操作序列,具有一个入口和一个出口点。

\item
边:代表从一个块到另一个块的控制流,包括条件分支和无条件分支。

\item
开始和结束节点:每个CFG都有一个唯一的入口节点和一个或多个出口节点。
\end{itemize}

例如,考虑我们之前用作示例的计算两个整数中最大值的函数;见图2.5:

\begin{cpp}
int max(int a, int b) {
  if (a > b)
    return a;
  return b;
}
\end{cpp}

\begin{center}
图6.1:CFG示例C++代码: max.cpp
\end{center}

相应的CFG可以表示如下:

\myGraphic{0.5}{content/part2/chapter6/images/2.png}{图6.2:max.cpp的CFG示例}

图6.2中显示的图表以视觉方式表示了max函数(来自图6.1)的CFG,由一系列连接的节点和有向边组成:


\begin{itemize}
\item
入口节点:顶部有一个"entry"节点,代表函数执行的起始点。

\item
条件节点:在入口节点下方,有一个标记为"a > b"的节点。这个节点代表函数中的条件语句,其中a和b进行比较。

\item
真和假条件的分支:
\begin{itemize}
\item
在真分支(左侧),有一个标记为"Return a"的节点,通过一条从"a > b"节点连接的边。这条边标记为"true",表示如果a大于b,则流程进入此节点。

\item
在假分支(右侧),有一个标记为"Return b"的节点,通过一条从"a > b"节点连接的边。这条边标记为"false",表示如果a不大于b,则流程进入此节点。
\end{itemize}

\item
出口节点:在"Return a"和"Return b"节点下方,汇聚于一点,有一个"exit"节点。这代表函数的终止点,即在返回a或b后控制流退出函数的点。
\end{itemize}

这个CFG有效地展示了max函数如何处理输入并根据比较做出决定返回哪个值。

CFG表示还可以用于估计函数的复杂性。简而言之,更复杂的图对应于更复杂的系统。我们将使用一个精确的复杂度定义,称为循环复杂度(或M\footnote{Thomas J. McCabe. A complexity measure. IEEE Transactions on Software Engineering, SE-2(4):308–320, 1976. ISSN 0098-5589. doi: 10.1109/TSE.1976.233837.}),可以按照以下方式计算:

$M = E - N + 2P$

其中:

\begin{itemize}
\item
E是图中的边数

\item
N是图中的节点数

\item
P是连接组件数(对于单个CFG,P通常为1)
\end{itemize}

对于前面讨论的max函数,CFG可以分析如下:

\begin{itemize}
\item
节点(N):有五个节点(Entry,a > b,Return a,b,Exit)

\item
边(E):有五条边(从Entry到a > b,从a > b到Return a,从a > b到Return b,从Return a到Exit,以及从Return b到Exit)

\item
连接组件(P):因为它是单个函数,P = 1
\end{itemize}

将这些值代入公式,我们得到以下结果:

$ M = 5 − 5 + 2 × 1 = 2$

因此,根据给定的CFG,max函数的循环复杂度为2。这表明代码中有两条线性独立的路径,对应于if语句的两个分支。

我们的下一步将是创建一个使用CFG计算循环复杂度的Clang-Tidy检查。





































