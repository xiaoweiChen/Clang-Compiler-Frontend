CFG是Clang工具进行高级静态分析的基本数据结构。Clang从其AST为函数构建CFG,识别基本块和控制流边。Clang的CFG构建处理各种C/C++构造,包括循环、条件语句、switch案例,以及setjmp/longjmp和C++异常等复杂构造。考虑使用图6.1的示例来研究这个过程。

\mySubsubsection{6.5.1.}{CFG构建示例}

图6.1的示例具有五个节点,如图6.2所示。我们将使用调试器来调查该过程:

\begin{shell}
$ lldb <...>/llvm-project/install/bin/clang-tidy --                   \
  -checks="-*,misc-cyclomaticcomplexity"                              \
  -config="{CheckOptions:                                             \
           [{key: misc-cyclomaticcomplexity.Threshold, value: '1'}]}" \
  max.cpp                                                             \
  -- -std=c++17 -Wno-all
\end{shell}

\begin{center}
图6.10:使用调试器调查CFG创建过程的演示
\end{center}

使用了与图6.8相同的命令,但将第一行命令更改为通过调试器运行检查。还更改了最后一行,以抑制编译器生成的所有警告。

\begin{myNotic}{重要提示}
高级静态分析是语义分析的一部分。如果Clang检测到不可达代码,将打印警告,由-Wunreachable-code选项控制。检测器是Clang语义分析的一部分,并使用CFG以及AST作为基本数据结构来检测此类问题。可以通过指定特殊的-Wno-all命令行选项来抑制这些警告,从而在Clang中禁用CFG初始化,该选项抑制由编译器生成的所有警告。
\end{myNotic}

我们将断点设置在CFGBuilder::createBlock函数上,该函数创建CFG块。

\begin{shell}
$ lldb <...>/llvm-project/install/bin/clang-tidy --                   \
  -checks="-*,misc-cyclomaticcomplexity"                              \
  -config="{CheckOptions:                                             \
           [{key: misc-cyclomaticcomplexity.Threshold, value: '1'}]}" \
  max.cpp                                                             \
  -- -std=c++17 -Wno-all
...
(lldb) b CFGBuilder::createBlock
Breakpoint 1: where = ...CFGBuilder::createBlock(bool) const ...
\end{shell}

\begin{center}
图6.11:运行调试器并为CFGBuilder::createBlock设置断点
\end{center}

如果运行调试器,会看到该函数调用了五次,所以max函数创建了五个CFG块:

\begin{shell}
1  (lldb) r
2  ...
3     frame #0: ...CFGBuilder::createBlock...
4     1690 /// createBlock - Used to lazily create blocks that are connected
5     1691 ///  to the current (global) successor.
6     1692 CFGBlock *CFGBuilder::createBlock(bool add_successor) {
7  -> 1693   CFGBlock *B = cfg->createBlock();
8     1694   if (add_successor && Succ)
9     1695     addSuccessor(B, Succ);
10    1696   return B;
11
12 (lldb) c
13 ...
14 (lldb) c
15 ...
16 (lldb) c
17 ...
18 (lldb) c
19 ...
20 (lldb) c
21 ...
22 1  warning generated.
23 max.cpp:1:5: warning: function 'max' has high cyclomatic complexity (2) [misc-cyclomaticcomplexity]
24 int max(int a, int b) {
25     ^
26 Process ... exited with status = 0 (0x00000000)
\end{shell}

\begin{center}
图6.12:创建CFG块,断点已显示
\end{center}

图6.12中显示的调试器演示可以视为CFG创建过程的入口点。现在,是时候深入探讨实现细节了。

\mySubsubsection{6.5.2.}{控制流图构造实现细节}

块是以逆序创建的,如图 6.13 所示。首先创建的是退出块,如图 6.13 第 4 行所示。CFG 构造器遍历作为参数传递的 clang::Stmt 对象(第 9 行),入口块最后创建(第 12 行):

\begin{cpp}
std::unique_ptr<CFG> CFGBuilder::buildCFG(const Decl *D, Stmt *Statement) {
  ...
  // Create an empty block that will serve as the exit block for the CFG.
  Succ = createBlock();
  assert(Succ == &cfg->getExit());
  Block = nullptr;  // the EXIT block is empty.  ...
  ...
  // Visit the statements and create the CFG.
  CFGBlock *B = Visit(Statement, ...);
  ...
  // Create an empty entry block that has no predecessors.
  cfg->setEntry(createBlock());
  ...
  return std::move(cfg);
}
\end{cpp}

\begin{center}
图 6.13: 从 clang/lib/Analysis/CFG.cpp 中简化的 buildCFG 实现
\end{center}

访问器使用 clang::Stmt::getStmtClass 方法,实现一个基于语句类型的自定义访问器:

\begin{cpp}
CFGBlock *CFGBuilder::Visit(Stmt * S, ...) {
  ...
  switch (S->getStmtClass()) {
  ...
  case Stmt::CompoundStmtClass:
  return VisitCompoundStmt(cast<CompoundStmt>(S), ...);
  ...
  case Stmt::IfStmtClass:
  return VisitIfStmt(cast<IfStmt>(S));
  ...
  case Stmt::ReturnStmtClass:
  ...
  return VisitReturnStmt(S);
  ...
  }
}
\end{cpp}


\begin{center}
图 6.14:语句访问器实现;用于我们的示例的案例已突出显示,代码取自 clang/lib/Analysis/CFG.cpp
\end{center}

示例包含两个返回语句和一个 if 语句,组合成一个复合语句。访问器的相关部分如图 6.14 所示。

本例中,传递的语句是一个复合语句;因此,图 6.14 的第 6 行激活。然后执行以下代码:

\begin{cpp}
CFGBlock *CFGBuilder::VisitCompoundStmt(CompoundStmt *C, ...) {
  ...
  CFGBlock *LastBlock = Block;

  for (Stmt *S : llvm::reverse(C->body())) {
    // If we hit a segment of code just containing ';' (NullStmts), we can
    // get a null block back.  In such cases, just use the LastBlock
    CFGBlock *newBlock = Visit(S, ...);

    if (newBlock)
      LastBlock = newBlock;

    if (badCFG)
      return nullptr;
    ...
  }

  return LastBlock;
}
\end{cpp}


\begin{center}
图 6.15:复合语句访问器,代码取自 clang/lib/Analysis/CFG.cpp
\end{center}

在示例创建 CFG 的过程中,访问了多个构造。第一个是 clang::IfStmt:

\begin{cpp}
CFGBlock *CFGBuilder::VisitIfStmt(IfStmt *I) {
  ...
  // Process the true branch.
  CFGBlock *ThenBlock;
  {
    Stmt *Then = I->getThen();
    ...
    ThenBlock = Visit(Then, ...);
    ...
  }

  // Specially handle "if (expr1 || ...)" and "if (expr1 && ...)"
  // ...
  if (Cond && Cond->isLogicalOp())
    ...
  else {
    // Now create a new block containing the if statement.
    Block = createBlock(false);
    ...
  }
  ...
}
\end{cpp}

\begin{center}
图 6.16: if 语句访问器,代码取自 clang/lib/Analysis/CFG.cpp
\end{center}

第 18 行为 if 语句创建了一个特殊块,我们也访问了第 8 行的"then"条件。

"then"条件访问一个返回语句:

\begin{cpp}
CFGBlock *CFGBuilder::VisitReturnStmt(Stmt *S) {
  // Create the new block.
  Block = createBlock(false);
  ...
  // Visit children
  if (ReturnStmt *RS = dyn_cast<ReturnStmt>(S)) {
    if (Expr *O = RS->getRetValue())
      return Visit(O, ...);
    return Block;
  }
  ...
}
\end{cpp}

\begin{center}
图 6.17:返回语句访问器,代码取自 clang/lib/Analysis/CFG.cpp
\end{center}

对于我们的示例,在第 3 行创建了一个块,并在第 8 行访问了返回表达式。返回表达式是一个简单的表达式,不需要创建新的块。

图 6.13 至图 6.17 中呈现的代码仅展示了块创建过程。为了简化,省略了一些重要的部分。构建过程包括以下内容:

\begin{itemize}
\item
边的创建: 典型的块可以有一个或多个后继者。每个块的节点列表及其后继者(边)列表维护了整个图形结构,表示符号程序执行。

\item
存储元信息: 每个块存储与其相关的额外元信息。例如,每个块保存了一组块内的语句。

\item
处理特殊情况: C++ 是一种复杂的语言,有许多不同的语言结构需要特殊处理。
\end{itemize}

控制流图(CFG)是高级代码分析的基本数据结构。Clang 使用 CFG 创建了几个工具。让我们简要地了解一下它们。

















