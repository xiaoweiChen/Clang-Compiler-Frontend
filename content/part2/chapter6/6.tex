
As mentioned earlier, the CFG is foundational for other analysis tools in Clang, several of which have been created atop the CFG. These tools also employ advanced mathematics to analyze various cases. The most notable tools are as follows \footnote{Kristóf Umann. A Survey of Dataflow Analyses in Clang. October 2020. URL \url{https://lists.llvm.org/pipermail/cfe-dev/2020-October/066937.html}.}:

\begin{itemize}
\item
LivenessAnalysis: Determines whether a computed value will be used before being overwritten, producing liveness sets for each statement and CFGBlock

\item
UninitializedVariables: Identifies the use of uninitialized variables through multiple passes, including initial categorization of statements and subsequent calculation of variable usages

\item
Thread Safety Analysis: Analyzes annotated functions and variables to ensure thread safety
\end{itemize}

LivenessAnalysis in Clang is essential for optimizing code by determining whether a value computed at one point will be used before being overwritten. It produces liveness sets for each statement and CFGBlock, indicating potential future use of variables or expressions. This backward "may" analysis simplifies read/write categorization by treating variable declarations and assignments as writes, and other contexts as reads, regardless of aliasing or field usage. Valuable in dead code elimination and compiler optimizations, such as efficient register allocation, it helps free up memory resources and improve program efficiency. Despite challenges with corner cases and documentation, its straightforward mplementation and the ability to cache and query results make it a vital tool in enhancing software performance and resource management.

\begin{myNotic}{Important note}
Forward analysis is a method used in programming to check how data moves through a program from start to finish. Following the data path step by step as the program runs allows us to see how it changes or where it goes. This method is instrumental for identifying issues such as improperly set-up variables or tracking data flow in the program. It contrasts with backward analysis, which starts at the end of the program and works backward.
\end{myNotic}

UninitializedVariables analysis in Clang, designed to detect the use of variables before initialization, operates as a forward "must" analysis. It involves multiple passes, including initial code scanning for statement classification and subsequent use of a fix-point algorithm to propagate information through the CFG. Handling more sophisticated scenarios than LivenessAnalysis, it faces challenges such as lacking support for record fields and non-reusable analysis results, limiting its efficiency in certain situations.

Thread Safety Analysis in Clang, a forward analysis, focuses on ensuring proper synchronization in multithreaded code. It computes sets of locked mutexes for each statement in a block and utilizes annotations to indicate guarded variables or functions. Translating Clang expressions into TIL (Typed Intermediate Language) \footnote{Kristóf Umann. A Survey of Dataflow Analyses in Clang. October 2020. URL \url{https://lists.llvm.org/pipermail/cfe-dev/2020-October/066937.html}.}, it effectively handles the complexity of C++ expressions and annotations. Despite strong C++ support and a sophisticated understanding of variable interactions, it faces limitations, such as lack of support for aliasing, which can lead to false positives.









