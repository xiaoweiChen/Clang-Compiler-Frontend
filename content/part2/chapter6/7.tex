
使用 Clang 的抽象语法树(AST)和控制流图(CFG)进行分析有一些局限性。最显著的局限性如下所述\footnote{Bruno Cardoso Lopes and Nathan Lanza. [RFC] An MLIR based Clang IR (CIR). June 2022. URL \url{https://discourse.llvm.org/t/rfc-an-mlir-based-clang-ir-cir/63319}}:


\begin{itemize}
\item
Clang的AST局限性:Clang的AST不适合进行数据流分析和控制流推理,因为丢失了重要的语言信息,导致了结果的不准确性和分析的低效。分析的健全性也是一个考虑因素,某些分析,如活性分析,如果足够精确而不是总是保守的,其精确性将非常有价值。

\item
Clang的CFG问题:虽然Clang的CFG旨在弥合AST和LLVM IR之间的差距,但它遇到了已知的问题,具有有限的跨过程能力,并且缺乏足够的测试覆盖。
\end{itemize}


\footnote{Bruno Cardoso Lopes和Nathan Lanza. [RFC] 基于MLIR的Clang IR (CIR). 2022年6月. URL \url{https://discourse.llvm.org/t/rfc-an-mlir-based-clang-ir-cir/63319}.}中提到的一个例子是与C++20中引入的新特性——C++协程相关。这个功能的某些方面是在Clang前端之外实现的,使用Clang的AST和CFG等工具无法看到。这一限制使得对这种功能的分析,尤其是生命周期分析,变得复杂。

尽管存在这些局限性,Clang的CFG仍然是一个在编译器和编译器工具开发中广泛使用的强大工具。还有其他工具正在积极开发中\footnote{Bruno Cardoso Lopes. [RFC] 将ClangIR上游化. 2024年1月. URL \url{https://discourse.llvm.org/t/rfc-upstreaming-clangir/76587}.},旨在弥补CFG功能上的不足。
