我们将创建一个Clang工具来帮助我们为用于单元测试的类重命名方法。首先,我们会描述一下 clang::Rewriter 类——这是用于代码修改的基本类。


\mySubsubsection{7.2.1.}{Clang中的代码修改支持}

clang::Rewriter 是Clang库中的一个类,它为在一个编译单元内的源代码提供重写操作的支持。它提供了在源代码的抽象语法树(AST)中插入、删除和替换代码的方法。开发人员可以使用 clang::Rewriter 进行复杂的代码修改,例如重构或生成新的代码结构。它可以应用于代码生成和代码重构任务,因此对于各种代码转换目的都非常有用。

该类有几个文本插入的方法;例如,clang::Rewriter::InsertText 在指定的源位置插入文本,并且 clang::SourceLocation 用来指定缓冲区中的确切位置,具体参见第4.4.1节《源管理器与源位置》。除了文本插入外,你还可以使用 clang::Rewriter::RemoveText 删除文本,或者使用 clang::Rewriter::ReplaceText 替换文本。后两种方法使用源范围(clang::SourceRange)来指定要删除或替换的文本的位置。

clang::Rewriter 使用 clang::SourceManager 访问需要修改的源代码,这在第4.4.1节《源管理器与源位置》中有详细说明。让我们来看看 Rewriter 如何在实际项目中使用。

\mySubsubsection{7.2.2.}{测试类}

假设我们有一个用于测试的类。类名以 "Test" 开头(例如 TestClass),但该类的公共方法没有使用 'test\_' 前缀。例如,该类有一个名为 pos 的公共方法(TestClass::pos),而不是 test\_pos(TestClass::test\_pos())。我们想要创建一个工具来为这些类的方法添加这样的前缀。

\begin{cpp}
class TestClass {
public:
  TestClass(){};
  void pos(){};

private:
  void private_pos(){};
};
\end{cpp}

原始代码

\begin{cpp}
class TestClass {
public:
  TestClass(){};
  void test_pos(){};

private:
  void private_pos(){};
};
\end{cpp}

修改后的代码

\begin{center}
图 7.1: 对 TestClass 的代码转换
\end{center}


因此,我们希望将方法 TestClass::pos(参见图 7.1)替换为 TestClass::test\_pos 在类声明中。

如果我们有一个调用该方法的代码,应该进行以下替换:

\begin{cpp}
TestClass test;
test.pos();
\end{cpp}

原始代码

\begin{cpp}
TestClass test;
test.test_pos();
\end{cpp}

修改后的代码


\begin{center}
图 7.2: 对 TestClass 方法调用的代码转换
\end{center}

该工具还应忽略所有已经应用了所需修改的公共方法,无论是手动还是自动完成的。换句话说,如果一个方法已经有了所需的 'test\_' 前缀,那么工具就不应该对其进行修改。

我们将创建一个名为 methodrename 的Clang工具,执行所有必需的代码修改。这个工具将利用第3.4节《递归AST访问者》中讨论的递归AST访问者。最重要的部分是实现 Visitor 类。让我们详细探讨它。

\mySubsubsection{7.2.3.}{Visitor 类实现}

我们的 Visitor 类应处理以下特定的 AST 节点:

clang::CXXRecordDecl:这涉及处理名称以 "Test" 开头的 C++ 类定义。对于这类类,所有用户定义的公共方法都应加上 "test\_" 前缀。

clang::CXXMemberCallExpr:此外,我们需要识别所有使用修改后方法的地方,并根据类定义中的方法重命名进行相应的更改。

对于 clang::CXXRecordDecl 节点的处理如下:

\begin{cpp}
bool VisitCXXRecordDecl(clang::CXXRecordDecl *Class) {
  if (!Class->isClass())
    return true;
  if (!Class->isThisDeclarationADefinition())
    return true;
  if (!Class->getName().starts_with("Test"))
    return true;
  for (const clang::CXXMethodDecl *Method : Class->methods()) {
    clang::SourceLocation StartLoc = Method->getLocation();
    if (!processMethod(Method, StartLoc, "Renamed method"))
      return false;
  }
  return true;
}
\end{cpp}

\begin{center}
图 7.3: CXXRecordDecl 访问者的实现
\end{center}

图 7.3 中的第 11 至 16 行代表了对所检查节点的要求。例如,相应的类名应该以 "Test" 开头(参见图 7.3 中的第 15 至 16 行),这里我们使用了 llvm::StringRef 类的 starts\_with() 方法。

验证这些条件后,我们继续检查找到的类中的方法。

验证过程实现在 Visitor::processMethod 方法中,其实现如下所示:

\begin{cpp}
bool processMethod(const clang::CXXMethodDecl *Method,
                   clang::SourceLocation StartLoc, const char *LogMessage) {
  if (Method->getAccess() != clang::AS_public)
    return true;
  if (llvm::isa<clang::CXXConstructorDecl>(Method))
    return true;
  if (!Method->getIdentifier() || Method->getName().starts_with("test_"))
    return true;

  std::string OldMethodName = Method->getNameAsString();
  std::string NewMethodName = "test_" + OldMethodName;
  clang::SourceManager &SM = Context.getSourceManager();
  clang::tooling::Replacement Replace(SM, StartLoc, OldMethodName.length(),
                                      NewMethodName);
  Replaces.push_back(Replace);
  llvm::outs() << LogMessage << ": " << OldMethodName << " to "
               << NewMethodName << "\n";
  return true;
}
\end{cpp}

\begin{center}
图 7.4: processMethod 实现
\end{center}

图 7.4 中的第 46 至 51 行包含了所需条件的检查。例如,在第 46 至 47 行,我们验证该方法是否为公共方法。第 48 至 49 行用于排除构造函数的处理,而第 50 至 51 行则用于排除已有所需前缀的方法。

主要的替换逻辑实现在第 53 至 58 行。特别是,在第 56 至 57 行,我们创建了一个特殊的 clang::tooling::Replacement 对象,它作为所需代码修改的包装器。该对象的参数如下:

\begin{enumerate}
\item
clang::SourceManager:我们在第 55 行从 clang::ASTContext 获取源管理器。

\item
clang::SourceLocation:源位置指定了替换的起始位置。该位置作为我们 processMethod 方法的第二个参数传递,如第 45 行所示。

\item
unsigned: 被替换文本的长度。

\item
clang::StringRef: 替换文本,我们在第 54 行创建它。
\end{enumerate}

我们在Visitor类中存储替换操作,这是一个私有成员:

\begin{cpp}
private:
  clang::ASTContext &Context;
  std::vector<clang::tooling::Replacement> Replaces;
There is a special getter to access the object outside the Visitor class:
  const std::vector<clang::tooling::Replacement> &getReplacements() {
  return Replaces;
}
\end{cpp}

我们在第59-60行记录操作,使用LogMessage作为日志消息的前缀。不同的AST节点使用不同的日志消息;例如,对于clang::CXXRecordDecl,我们使用"Renamed method"(见图7.3,第19行)。

对于方法调用,日志消息将有所不同。相应的处理如下所示:

\begin{cpp}
bool VisitCXXMemberCallExpr(clang::CXXMemberCallExpr *Call) {
  if (clang::CXXMethodDecl *Method = Call->getMethodDecl()) {
    clang::CXXRecordDecl *Class = Method->getParent();
    if (!Class->getName().starts_with("Test"))
      return true;
    clang::SourceLocation StartLoc = Call->getExprLoc();
    return processMethod(Method, StartLoc, "Renamed method call");
   }
   return true;
}
\end{cpp}

\begin{center}
图7.5:CXXMemberCallExpr访问者实现
\end{center}

我们在第27-29行验证持有测试方法的类名是否以'Test'前缀开头。在第30行获取替换的源位置。在第31行,我们调用我们的processMethod函数来处理找到的方法,将"Renamed method call"作为日志消息传递给调用。

Consumer类初始化Visitor并在HandleTranslationUnit方法中开始AST遍历。该类可以编写如下:

\mySubsubsection{7.2.4.}{Consumer类实现}

Consumer类初始化Visitor,并在HandleTranslationUnit方法中开始AST遍历。类可以这样编写:

\begin{cpp}
class Consumer : public clang::ASTConsumer {
public:
  void HandleTranslationUnit(clang::ASTContext &Context) override {
    Visitor V(Context);
    V.TraverseDecl(Context.getTranslationUnitDecl());

    // Apply the replacements.
    clang::Rewriter Rewrite(Context.getSourceManager(), clang::LangOptions());
    auto &Replaces = V.getReplacements();
    for (const auto &Replace : Replaces) {
      if (Replace.isApplicable()) {
        Replace.apply(Rewrite);
      }
    }

    // Apply the Rewriter changes.
    if (Rewrite.overwriteChangedFiles()) {
      llvm::errs() << "Error: Cannot apply changes to the file\n";
    }
  }
};
} // namespace methodrename
\end{cpp}

\begin{center}
图7.6:Consumer类实现
\end{center}

我们初始化Visitor并在第9-10行开始遍历(见图7.6)。在第13行创建Rewriter,并在第14-19行应用替换。最后,在第22-24行将结果存储在原始文件中。

Visitor和Consumer类被包装在clangbook::methodrename命名空间中。Consumer实例在FrontendAction类中创建。这个类的实现与图3.8中详细描述的RecursiveVisitor和DeclVisitor相似。唯一的区别是在新工具中使用clangbook::methodrename命名空间。

\mySubsubsection{7.2.5.}{构建配置和主函数}

我们的工具的主函数与图3.21中定义的递归访问者类似:

\begin{cpp}
int main(int argc, const char **argv) {
  llvm::Expected<clang::tooling::CommonOptionsParser> OptionsParser =
    clang::tooling::CommonOptionsParser::create(argc, argv, TestCategory);
  if (!OptionsParser) {
    llvm::errs() << OptionsParser.takeError();
    return 1;
  }
  clang::tooling::ClangTool Tool(OptionsParser->getCompilations(),
                                 OptionsParser->getSourcePathList());
   return Tool.run(clang::tooling::newFrontendActionFactory<
                   clangbook::methodrename::FrontendAction>()
                   .get());
}
\end{cpp}

\begin{center}
图7.7:'methodrename'测试工具的主函数
\end{center}

如您所见,我们只在第23行更改了自定义前端动作的命名空间名称。

构建配置如下指定:

\begin{cmake}
cmake_minimum_required(VERSION 3.16)
project("methodrename")

if ( NOT DEFINED ENV{LLVM_HOME})
  message(FATAL_ERROR "$LLVM_HOME is not defined")
else()
  message(STATUS "$LLVM_HOME found: $ENV{LLVM_HOME}")
  set(LLVM_HOME $ENV{LLVM_HOME} CACHE PATH "Root of LLVM installation")
  set(LLVM_LIB ${LLVM_HOME}/lib)
  set(LLVM_DIR ${LLVM_LIB}/cmake/llvm)
  find_package(LLVM REQUIRED CONFIG)
  include_directories(${LLVM_INCLUDE_DIRS})
  link_directories(${LLVM_LIBRARY_DIRS})
  set(SOURCE_FILE MethodRename.cpp)
  add_executable(methodrename ${SOURCE_FILE})
  set_target_properties(methodrename PROPERTIES COMPILE_FLAGS "-fno-rtti")
  target_link_libraries(methodrename
    LLVMSupport
    clangAST
    clangBasic
    clangFrontend
    clangSerialization
    clangToolingCore
    clangRewrite
    clangTooling
  )
endif()
\end{cmake}

\begin{center}
图7.8:构建'methodrename'测试工具的配置
\end{center}

相比于图3.20中的代码,最显著的变化是在第23和24行,我们添加了两个新的库来支持代码修改:clangToolingCore和clangRewrite。其他变化包括工具的新名称(第2行)以及包含主函数的源文件(第14行)。

我们完成了代码,就该构建并运行我们的工具了。

\mySubsubsection{7.2.6.}{运行代码修改工具}

程序可以使用我们在第3.3节AST遍历中使用的相同命令序列进行编译,见图3.11:

\begin{shell}
export LLVM_HOME=<...>/llvm-project/install
mkdir build
cd build
cmake -G Ninja -DCMAKE_BUILD_TYPE=Debug ...
ninja
\end{shell}

\begin{center}
图7.9:'methodrename'工具的配置和构建命令
\end{center}

我们可以对以下测试文件(TestClass.cpp)运行创建的工具:

\begin{cpp}
class TestClass {
public:
  TestClass(){};
  void pos(){};
};

int main() {
  TestClass test;
  test.pos();
  return 0;
}
\end{cpp}


\begin{center}
图7.10:原始的TestClass.cpp
\end{center}

我们可以按以下方式运行工具:

\begin{shell}
$ ./methodrename TestClass.cpp -- -std=c++17
Renamed method: pos to test_pos
Renamed method call: pos to test_pos
\end{shell}


\begin{center}
图7.11:在TestClass.cpp上运行methodrename Clang工具
\end{center}

我们可以看到,方法TestClass::pos重命名为TestClass::test\_pos。方法调用也得到了更新,如下所示:

\begin{cpp}
class TestClass {
public:
  TestClass(){};
  void test_pos(){};
};

int main() {
  TestClass test;
  test.test_pos();
  return 0;
}
\end{cpp}

\begin{center}
图7.12:修改后的TestClass.cpp
\end{center}

提供的示例演示了Clang如何协助创建重构工具。创建的Clang工具使用递归访问者来设置所需的代码转换。另一个可能的选项是使用我们在第5章Clang-Tidy检查框架中研究过的Clang-Tidy。让我们更详细地探讨这个选项。




















































