集成开发环境(IDE)是一种软件应用程序或平台,它提供了一套全面的工具和特性,以帮助开发人员创建、编辑、调试和管理软件代码。IDE通常包括具有语法高亮显示的代码编辑器、调试功能、项目管理特性、版本控制集成,以及通常支持各种编程语言和框架的插件或扩展。

流行的IDE示例包括Visual Studio/VS Code、IntelliJ IDEA、Emacs和Vim。这些工具旨在简化开发过程,使开发人员能够更高效地编写、测试和维护代码。

典型的IDE支持多种语言,并且将每种语言集成进去都是一个具有挑战性的任务。每种语言都需要特定的支持,这在图8.1中可以可视化。值得注意的是,不同编程语言的开发过程有许多相似之处。例如,图8.1中显示的语言都包含一个代码导航功能,允许开发人员快速定位和查看代码库中符号或标识符的定义。

\myGraphic{0.5}{content/part2/chapter8/images/1.png}{图8.1:IDE中的编程语言集成}

这个功能在本章中将被称为“跳转到定义”。这些相似之处表明,可以通过引入一个中间层,即语言服务器协议(LSP),来简化图8.1中的关系。

\myGraphic{0.8}{content/part2/chapter8/images/2.png}{图8.2:使用LSP在IDE中集成编程语言}

LSP项目于2015年由微软发起,作为其改进轻量级、开源代码编辑器VS Code的一部分。微软意识到需要一种标准化的方式,在VS Code和其他代码编辑器中为不同编程语言提供丰富的语言服务。

LSP迅速在开发者社区中流行并得到采用。许多代码编辑器和IDE,包括VS Code、Emacs和Eclipse,开始实现对LSP的支持。

各种编程语言的语言服务器实现出现了。这些语言服务器由微软和开源社区开发,提供了特定语言的智能和特性,使得将语言特性集成到不同的编辑器中变得更加容易。

在本章中,我们将探索Clangd,它是clang-tools-extra的一部分语言服务器。Clangd利用Clang编译器前端,并提供了一系列的代码分析和语言支持特性。Clangd帮助开发人员完成智能代码、语义分析和实时诊断,帮助他们更高效地编写代码,并在开发过程中尽早发现错误。我们将详细探讨Clangd,从IDE(VS Code)和Clangd之间的实际交互示例开始。我们将从环境设置开始,包括Clangd构建和VS Code设置。
































