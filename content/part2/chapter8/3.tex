我们将开始通过构建Clangd来设置我们的环境。然后,安装VS Code,设置Clangd扩展,并在其中配置Clangd。

\mySubsubsection{8.3.1.}{构建Clangd}

在发布模式下构建Clangd,就像我们在第1.3.3节中为LLDB所做的那样。这是因为性能对于IDE至关重要。例如,Clangd需要构建一个AST来提供代码导航功能。如果用户修改了一个文档,该文档应该重建,导航功能在重建过程完成之前是不可用的。这可能导致IDE响应延迟。为了防止IDE响应缓慢,我们应该确保Clangd在构建时包含了所有所需的优化。可以使用以下项目配置命令:

\begin{shell}
cmake -G Ninja -DCMAKE_BUILD_TYPE=Release -DCMAKE_INSTALL_PREFIX=../install -DLLVM_TARGETS_TO_BUILD="X86" -DLLVM_ENABLE_PROJECTS="clang;clang-tools-extra" ../llvm
\end{shell}

\begin{center}
图8.3:Clangd的发布配置构建
\end{center}

该命令需要从我们在第1.3.3节中创建的发布文件夹运行,如图1.8所示。如图8.3所示,我们已经启用了两个项目:clang和clang-tools-extra。

可以使用以下命令构建和安装Clangd:

\begin{shell}
$ ninja install-clangd -j $(nproc)
\end{shell}

此命令将利用系统上可用的最大线程数,并将二进制文件安装到图8.3中的CMake命令中指定的文件夹中,该文件夹位于LLVM源树下的安装文件夹中。

构建Clangd二进制文件后,下一步包括安装VS Code并配置它与Clangd一起工作。

\mySubsubsection{8.3.2.}{VS Code安装和设置}

可以从VS Code网站下载并安装VS Code: \url{https://code.visualstudio.com/download}.

运行VS Code后的第一步是安装Clangd扩展,开源扩展可用于通过LSP与Clangd一起工作。该扩展的源代码可以在GitHub上找到: https://github.com/clangd/vscode-clangd。可以直接从VS Code内部轻松安装最新版本的扩展。

要做到这一点,请按Ctrl+Shift+X(或对于macOS,\includegraphics[width=0.02\textwidth]{content/part2/chapter8/images/3.png}+Shift+X)打开扩展面板。搜索Clangd并点击安装按钮。

\myGraphic{0.7}{content/part2/chapter8/images/4.png}{图8.4:安装Clangd扩展}

安装扩展后,需要对其进行设置。主要步骤是指定Clangd可执行文件的路径。

可以通过File -> Preferences -> Settings菜单(或按Ctrl + ,,对于macOS,\includegraphics[width=0.02\textwidth]{content/part2/chapter8/images/3.png} +,)访问此设置,如图8.5所示:

\myGraphic{0.7}{content/part2/chapter8/images/5.png}{图8.5:设置Clangd扩展}

如图8.5所示,已经配置了Clangd路径为/home/ivanmurashko/clangbook/\\llvm-project/install/bin/clangd。这是第8.3.1节中用于安装Clangd二进制文件的路径。

可以打开您喜欢的C/C++源文件并尝试在其中导航。例如,可以搜索一个token的定义,在源文件和头文件之间切换等。下一个示例中,我们将调查如何通过LSP实现导航,特别是跳转到定义的功能。

\begin{myNotic}{重要提示}
我们的设置仅适用于不需要特殊编译标志的简单项目。如果项目需要特殊的配置才能构建,必须使用一个生成的compile\_commands.json文件,该文件应放置在项目的根目录下。该文件应包含JSON格式的编译数据库(CDB),指定项目中每个文件的编译标志。有关设置的更多信息,请参阅图9.5,Clangd大型项目的设置。
\end{myNotic}

安装了所需的组件之后,现在准备进行一个LSP演示,其中我们将模拟IDE中的典型开发活动(例如打开和修改文档、跳转到token定义等),并探索它是如何通过LSP表示的。











































































