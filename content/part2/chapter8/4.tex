这个简短的LSP演示中,我们将展示Clangd如何打开一个文件并查找一个符号的定义。Clangd拥有一个全面的日志子系统,提供了深入了解其与IDE交互的宝贵信息,将使用日志子系统来获取必要的信息。

\mySubsubsection{8.4.1.}{演示描述}

例子中,我们打开一个如以下屏幕截图所示的测试文件,并检索doPrivateWork标记的定义:

\myGraphic{0.7}{content/part2/chapter8/images/6.png}{图8.6:跳转到doPrivateWork标记的定义和悬停}

VS Code通过标准输入/输出与Clangd通信,使用Clangd日志来捕获信息。

这可以通过在VS Code设置中创建一个包装shell脚本,而不是使用实际的clangd二进制文件来实现:

\myGraphic{0.7}{content/part2/chapter8/images/7.png}{图8.7:在VS Code中设置包装shell脚本}

可以使用以下脚本,clangd.sh:

\begin{shell}
#!/bin/sh
$HOME/clangbook/llvm-project/install/bin/clangd -log verbose -pretty 2> /tmp/clangd.log
\end{shell}

\begin{center}
图8.8:用于clangd的包装shell脚本
\end{center}

图8.8中,使用了两个日志选项:

\begin{itemize}
\item
第一个,-log verbose,激活详细日志记录,以确保实际来自Clangd的LSP消息将被记录。

\item
第二个选项,-pretty,用于提供格式化的JSON消息。还将标准错误输出重定向到日志文件,在我们的例子中是/tmp/clangd.log。
\end{itemize}

结果,文件将包含示例演示的日志。可以使用以下命令查看这些日志:

\begin{shell}
$ cat /tmp/clangd.log
\end{shell}

日志中,可以找到由VS Code发送的"textDocument/definition":

\begin{shell}
V[16:24:39.336] <<< {
    "id": 13,
    "jsonrpc": "2.0",
    "method": "textDocument/definition",
    "params": {
        "position": {
            "character": 26,
            "line": 7
        },
        "textDocument": {
            "uri": "file:///home/ivanmurashko/clangbook/helper.hpp"
        }
    }
}
\end{shell}


\begin{center}
图8.9:由IDE发送的"textDocument/definition"请求
\end{center}

IDE发送的请求由Clangd接收并处理。相应的日志记录如下:

\begin{shell}
I[16:24:39.336] <-- textDocument/definition(13)
V[16:24:39.336] ASTWorker running Definitions on version 1 of /home/.../
helper.hpp
\end{shell}


\begin{center}
图8.10:Clangd中"textDocument/definition"请求的处理
\end{center}

最后,Clangd创建响应并将其发送到IDE。相应的日志记录显示回复已发送:

\begin{shell}
I[16:24:39.336] --> reply:textDocument/definition(13) 0 ms
V[16:24:39.336] >>> {
    "id": 13,
    "jsonrpc": "2.0",
    "result": [
    {
        "range": {
            "end": {
                "character": 20,
                "line": 10
            },
            "start": {
                "character": 7,
                "line": 10
            }
        },
        "uri": "file:///home/ivanmurashko/clangbook/helper.hpp"
    }
    ]
}
\end{shell}


\begin{center}
图8.11:来自Clangd的"textDocument/definition"回复
\end{center}

日志将是我们调查LSP内部的主要工具。

\mySubsubsection{8.4.2.}{LSP演示}

LSP演示包括与Clangd服务器的多个请求和响应。演示从"initialize"请求开始。然后,我们打开一个文档,VS Code发送一个"textDocument/didOpen"通知。在此请求之后,当打开的文件状态发生变化时,Clangd将定期响应"textDocument/publishDiagnostics"通知。例如,这种情况发生在编译完成并准备好处理导航请求时。接下来,为某个标记发起一个转到定义的请求,Clangd响应找到的定义的位置信息。我们还研究了Clangd如何处理客户端通过"textDocument/didChange"通知的文件修改。当我们关闭打开的文件时,我们通过"textDocument/didClose"请求结束演示。下面是一张描述图表:

\myGraphic{0.6}{content/part2/chapter8/images/8.png}{图8.12:LSP演示示例}

让我们详细看看这个例子。我们将从"initialize"请求开始。

\mySamllsection{初始化}

为了建立通信,客户端(代码编辑器或IDE)和语言服务器交换JSON-RPC消息。初始化过程从客户端向语言服务器发送一个"initialize"请求开始,指定它支持的功能。VS Code实际发送的请求相当大,下面是一个简化版本,其中请求的部分内容用"…"代替:

\begin{shell}
{
 "id": 0,
 "jsonrpc": "2.0",
 "method": "initialize",
 "params": {
   "capabilities": {
     ...
     "textDocument": {
       ...
       "definition": {
         "dynamicRegistration": true,
         "linkSupport": true
       },
       ...
     },
     "clientInfo": {
       "name": "Visual Studio Code",
       "version": "1.85.1"
     },
     ...
   }
 }
\end{shell}


\begin{center}
图8.13:VS Code到Clangd(初始化请求)
\end{center}

请求中,客户端(VS Code)告诉服务器(Clangd)客户端支持哪些功能;例如,图8.13的第10-13行,客户端表示它支持用于转到定义请求的"textDocument/definition"请求类型。

语言服务器用包含服务器支持的功能的响应来回复请求:

\begin{shell}
{
  "id": 0,
  "jsonrpc": "2.0",
  "result": {
    "capabilities": {
      ...
      "definitionProvider": true,
      ...
    },
    "serverInfo": {
      "name": "clangd",
      "version": "clangd version 16.0.6 (https://github.com/llvm/llvm-project.git 7cbf1a2591520c2491aa35339f227775f4d3adf6) linux x86_64-unknown-linux-gnu"
    }
  }
}
\end{shell}

\begin{center}
图8.14:Clangd到VS Code(初始化回复)
\end{center}

相同的id用于将请求与其回复连接起来,Clangd回复称它支持图8.14第7行中指定的转到定义请求。因此,客户端(VS Code)可以向服务器发送导航请求,将在后面的图8.19(转到定义)中探讨。

VS Code通过发送一个"initialized"通知来确认初始化:

\begin{shell}
{
  "jsonrpc": "2.0",
  "method": "initialized"
}
\end{shell}

与"initialize"请求相反,这是一个通知,它不期望服务器有任何响应。因此,没有"id"字段。"initialized"通知只能发送一次,并且应该在客户端发送其他请求或通知之前收到。初始化后,准备好打开一个文档并发送相应的"textDocument/didOpen"通知。

\mySamllsection{打开文档}

当开发者打开一个C++源文件时,客户端发送一个"textDocument/didOpen"通知以告知语言服务器关于新打开的文件。在我们的例子中,打开的文件位于/home/ivanmurashko/clangbook/helper.hpp,VS Code发送的通知如下所示:

\begin{shell}
{
  "jsonrpc": "2.0",
  "method": "textDocument/didOpen",
  "params": {
    "textDocument": {
      "languageId": "cpp",
      "text": "#pragma once\n\nnamespace clangbook {\nclass Helper {\npublic:\n  Helper(){};\n\n  void doWork() { doPrivateWork(); }\n\nprivate:\n  void doPrivateWork() {}\n};\n}; // namespace clangbook\n",
      "uri": "file:///home/ivanmurashko/clangbook/helper.hpp",
      "version": 1
     }
   }
}
\end{shell}


\begin{center}
图8.15:VS Code到Clangd (didOpen通知)
\end{center}

VS Code发送通知,其中包含在"params/ textDocument"字段中的参数。这些参数包括在"uri"字段中的文件名和在"text"字段中的源文件文本。

Clangd在接收到'didOpen'通知后开始编译文件,构建了一个抽象语法树(AST)并从中提取有关不同标记的语义信息。服务器使用此信息来区分具有相同名称的不同标记。例如,可以使用一个名为'foo'的标记,可能作为类成员或局部变量,具体取决于其使用的范围,如下面的代码所示:

\begin{cpp}
class TestClass {
public:
  int foo(){return 0};
};

int main() {
  TestClass test;
  int foo = test.foo();
  return foo;
}
\end{cpp}

\begin{center}
图8.16:foo.hpp中的'foo'标记的出现
\end{center}

如第8行所示,我们在两个地方使用了'foo'标记:作为函数调用和局部变量定义。

转到定义请求将在编译过程完成后延迟。值得注意的是,大多数请求被放入队列中,等待编译过程完成。该规则有一些例外,一些请求可以在不具有AST的情况下执行,并提供有限的提供功能,其中一个例子是代码格式化请求。代码格式化不需要AST,因此格式化功能可以在构建AST之前提供。

如果文件的状态发生变化,Clangd将使用"textDocument/publishDiagnostics"通知通知VS Code。当编译过程完成后,Clangd将向VS Code发送通知:

\begin{shell}
{
  "jsonrpc": "2.0",
  "method": "textDocument/publishDiagnostics",
  "params": {
    "diagnostics": [],
    "uri": "file:///home/ivanmurashko/clangbook/helper.hpp",
    "version": 1
  }
}
\end{shell}

\begin{center}
图8.17:Clangd到VS Code (publishDiagnostics通知)
\end{center}

没有编译错误;params/diagnostics为空。如果代码包含编译错误或警告,它将包含错误或警告描述,如图8.18所示。

\begin{shell}
{
  "jsonrpc": "2.0",
  "method": "textDocument/publishDiagnostics",
  "params": {
    "diagnostics": [
    {
      "code": "expected_semi_after_expr",
      "message": "Expected ';' after expression (fix available)",
      "range": {
        "end": {
          "character": 35,
          "line": 7
        },
        "start": {
          "character": 34,
          "line": 7
        }
      },
      "relatedInformation": [],
      "severity": 1,
      "source": "clang"
    }
    ],
    "uri": "file:///home/ivanmurashko/clangbook/helper.hpp",
    "version": 5
  }
}
\end{shell}

\begin{center}
图8.18:Clangd到VS Code (带有编译错误的publishDiagnostics)
\end{center}

VS Code处理诊断并显示,如下面的截图所示:

\myGraphic{0.7}{content/part2/chapter8/images/9.png}{图8.19:helper.hpp中的编译错误}

编译完成后,我们收到了"textDocument/publishDiagnostics",Clangd准备好处理导航请求,例如"textDocument/definition"(转到定义)。

\mySamllsection{转到定义}

为了在C++文件中找到符号的定义,客户端向语言服务器发送一个"textDocument/\\definition"请求:

\begin{shell}
{
  "id": 13,
  "jsonrpc": "2.0",
  "method": "textDocument/definition",
  "params": {
    "position": {
      "character": 26,
      "line": 7
    },
    "textDocument": {
       "uri": "file:///home/ivanmurashko/clangbook/helper.hpp"
    }
  }
}
\end{shell}

\begin{center}
图8.20:VS Code到Clangd (textDocument/definition请求)
\end{center}

编辑器中,行位置被指定为7而不是实际的8行,如图8.6所示。这是因为行编号从0开始。

语言服务器响应C++代码中的定义位置:

\begin{shell}
{
  "id": 13,
  "jsonrpc": "2.0",
  "result": [
  {
    "range": {
      "end": {
        "character": 20,
        "line": 10
      },
      "start": {
        "character": 7,
        "line": 10
      }
    },
    "uri": "file:///home/ivanmurashko/clangbook/helper.hpp"
  }
  ]
}
\end{shell}

\begin{center}
图8.21:Clangd到VS Code (textDocument/definition响应)
\end{center}

服务器响应了实际的定义位置。IDE中的另一个常见操作是文档修改。此功能由"textDocument/didChange"通知提供。

\mySamllsection{修改文档}

作为文档修改的一部分,在第6行插入一个注释,// 构造函数,如图8.22所示。

\myGraphic{0.7}{content/part2/chapter8/images/10.png}{图 8.22: 修改文档}

VS Code将检测到文档已修改,并通过以下通知通知LSP服务器(Clangd):

\begin{shell}
{
  "jsonrpc": "2.0",
  "method": "textDocument/didChange",
  "params": {
    "contentChanges": [
      {
        "range": {
          "end": {
            "character": 13,
            "line": 5
          },
          "start": {
            character": 13,
            "line": 5
          }
        },
        "rangeLength": 0,
        "text": "// Constructor"
      }
    ],
    "textDocument": {
      "uri": "file:///home/ivanmurashko/clangbook/helper.hpp",
      "version": 2
    }
  }
}
\end{shell}

\begin{center}
图8.23:VS Code到Clangd (didChange通知)
\end{center}

通知包含要替换文档中指定范围的文本和范围,通知的一个重要部分是"version"字段,指定了文档的版本。

可以观察到版本从1(用于文档打开,如图8.15所示的第9行)更改为2(用于文档修改,如图8.23所示的第23行)。

Clangd开始编译文档,因为文档修改可能导致结果AST发生重大变化,这用于导航请求。一旦编译完成,服务器将相应地响应"textDocument/publishDiagnostics"通知,如图8.24所示。

\begin{shell}
{
  "jsonrpc": "2.0",
  "method": "textDocument/publishDiagnostics",
  "params": {
    "diagnostics": [],
    "uri": "file:///home/ivanmurashko/clangbook/helper.hpp",
     "version": 2
}
\end{shell}


\begin{center}
图8.24:Clangd到VS Code (publishDiagnostics通知)
\end{center}

修改后的文档发送了诊断,它包含指向版本2的版本字段,这与修改后的文档对应,如图8.24所示的第7行。

我们的最后一个操作是关闭文档。

\mySamllsection{关闭文档}

当完成对文档的工作并关闭它时,VS Code向语言服务器发送一个"textDocument/\\didClose"通知:

\begin{shell}
{
  "jsonrpc": "2.0",
  "method": "textDocument/didClose",
  "params": {
    "textDocument": {
      "uri": "file:///home/ivanmurashko/clangbook/helper.hpp"
    }
  }
}
\end{shell}

\begin{center}
图8.25:VS Code到Clangd (textDocument/didClose请求)
\end{center}

收到请求后,Clangd将从其内部结构中移除文档。Clangd将不再发送文档的更新,因此将通过发送最终的空"textDocument/publishDiagnostics"消息来清空客户端(例如,VS Code中的问题窗格)显示的诊断列表,如图8.26所示。

\begin{shell}
{
  "jsonrpc": "2.0",
  "method": "textDocument/publishDiagnostics",
  "params": {
    "diagnostics": [],
    "uri": "file:///home/ivanmurashko/clangbook/helper.hpp"
  }
}
\end{shell}

\begin{center}
图8.26:Clangd到VS Code (textDocument/didClose请求)
\end{center}

所展示的例子演示了Clangd和VS Code之间的典型交互,提供的例子利用了Clang前端的功能,即基本Clang功能。另一方面,Clangd与Clang的其他工具(如Clang-Format和Clang-Tidy)有着紧密的联系,并可以重用这些工具提供的功能。
























