Clangd利用LLVM模块架构的优势,与其他Clang工具有着非常紧密的集成。特别是,Clangd使用Clang-Format库来提供格式化功能,并使用Clang-Tidy库(如带有clang-tidy检查的库)来在IDE中支持代码检查器。这种集成在下面的图中示意性地展示了出来:

\myGraphic{0.7}{content/part2/chapter8/images/11.png}{图8.27:带有LSP扩展和Clangd服务器的VS Code用于C++}

格式化配置来自.clang-format(参见第7.4.1节,Clang-Format配置和使用示例),而代码检查器的配置来自.clang-tidy(参见图5.12,Clang-Tidy配置)。让我们看看Clangd中格式化是如何工作的。


\mySubsubsection{8.5.1.}{Clangd使用LSP消息支持代码格式化}

Clangd为代码格式化提供了强大的支持。这个特性对于开发者在他们的C和C++项目中保持一致的代码风格和可读性至关重要。Clangd通过LSP消息实现此功能,主要是使用“textDocument/formatting”和“textDocument/rangeFormatting”请求。

\mySamllsection{格式化整个文档}

当开发者想要格式化文档的全部内容时,会使用“textDocument/formatting”请求。在VS Code中,这个请求通常是通过用户按下Ctrl + Shift + I(或者在macOS上是 \includegraphics[width=0.02\textwidth]{content/part2/chapter8/images/3.png} + Shift + I)来发起的;IDE会向Clangd发送一个针对整个文档的“textDocument/formatting”请求:

\begin{shell}
{
  "id": 9,
  "jsonrpc": "2.0",
  "method": "textDocument/formatting",
  "params": {
    "options": {
      "insertSpaces": true,
      "tabSize": 4
    },
    "textDocument": {
      "uri": "file:///home/ivanmurashko/clangbook/helper.hpp"
    }
  }
}
\end{shell}


\begin{center}
图8.28:VS Code向Clangd发送(textDocument/formatting请求)
\end{center}

Clangd通过使用项目中.clang-format文件指定的代码风格配置来处理这个请求。.clang-format文件包含了格式化规则和偏好设置,允许开发者定义他们期望的代码风格;详见第7.4.1节,Clang-Format配置和使用示例。

响应包含了要应用于打开文档的修改列表:

\begin{shell}
{
  "id": 9,
  "jsonrpc": "2.0",
  "result": [
  {
    "newText": "\n  ",
    "range": {
      "end": {
        "character": 0,
        "line": 5
      },
      "start": {
        "character": 7,
        "line": 4
      }
    }
  }
  ]
}
\end{shell}


\begin{center}
图8.29:Clangd向VS Code返回(textDocument/formatting响应)
\end{center}

在示例中,我们应该在图8.29中的第7-16行指定的范围内,用第6行指定的新文本替换文本。

\mySamllsection{格式化特定的代码范围}

除了格式化整个文档外,Clangd还支持在文档内格式化特定的代码范围。这是通过使用“textDocument/rangeFormatting”请求实现的。开发者可以在代码中选择一个范围,比如一个函数、一个代码块,甚至只是几行,然后请求对该特定范围的格式化,如下面的屏幕截图所示:

\myGraphic{0.7}{content/part2/chapter8/images/12.png}{图8.30:在helper.hpp中重新格式化特定的代码范围}

当选择菜单项或按下Ctrl + K然后Ctrl + F(或者在macOS上是 \includegraphics[width=0.02\textwidth]{content/part2/chapter8/images/3.png} + K然后 \includegraphics[width=0.02\textwidth]{content/part2/chapter8/images/3.png} + F),VS Code会向Clangd发送以下请求:

\begin{shell}
{
  "id": 89,
  "jsonrpc": "2.0",
  "method": "textDocument/rangeFormatting",
  "params": {
    "options": {
      "insertSpaces": true,
      "tabSize": 4
    },
    "range": {
      "end": {
        "character": 2,
        "line": 10
      },
      "start": {
        "character": 0,
        "line": 3
      }
    },
    "textDocument": {
      "uri": "file:///home/ivanmurashko/clangbook/helper.hpp"
    }
  }
}
\end{shell}

\begin{center}
图8.31:VS Code向Clangd发送(textDocument/rangeFormatting请求)
\end{center}

“textDocument/rangeFormatting"请求指定了要在文档中格式化的范围,Clangd将相同的格式化规则应用于这个特定的代码段,这些规则来自.clang-format文件。响应将与用于格式化请求的响应类似,并将包含应应用于原始文本的修改,如图8.29所示。唯一的区别将是方法名称,在这种情况下应该是"textDocument/rangeFormatting”。

通过Clangd集成的另一个工具是Clang-Tidy,它与刚刚描述的格式化功能相比,以不同的方式使用LSP协议。

\mySubsubsection{8.5.2.}{Clang-Tidy}

正如我们所见,Clangd使用特定的LSP方法来实现与Clang-Format的集成:

\begin{itemize}
\item
"textDocument/formatting"

\item
"textDocument/rangeFormatting"
\end{itemize}

另一方面,与Clang-Tidy的集成实现方式不同,它重用"publishDiagnostics"通知来报告代码检查器的警告和错误。

让我们探究一下它是如何工作的,并首先创建一个自定义的Clang-Tidy配置。

\mySamllsection{Clang-Tidy与LSP的集成}

我们将运行最近为测试方法重命名而创建的misc-methodrename检查,参见第7.3节,Clang-Tidy作为代码修改工具。我们的Clang-Tidy配置将如下所示:

\begin{shell}
Checks: '-*,misc-methodrename'
\end{shell}

\begin{center}
图8.32:用于IDE集成的.clang-tidy配置
\end{center}

带有配置的.clang-tidy文件应放置在我们的测试项目文件夹中。

如果我们把helper类重命名为TestHelper,我们将能够观察到在第7.3节,Clang-Tidy作为代码修改工具中创建的lint检查开始报告关于测试类使用的不正确方法名的诊断信息。相应的诊断信息显示在下拉窗格和PROBLEMS标签页中,如下面的屏幕截图所示:

\myGraphic{0.7}{content/part2/chapter8/images/13.png}{图8.33:Clang-Tidy集成}

消息作为诊断信息的一部分显示。具体来说,以下通知是从Clang发送到VS Code的:

\begin{shell}
{
  "jsonrpc": "2.0",
  "method": "textDocument/publishDiagnostics",
  "params": {
  "diagnostics": [
  {
    "code": "misc-methodrename",
    "codeDescription": {
      "href": "https://clang.llvm.org/extra/clang-tidy/checks/misc/
               methodrename.html"
    },
    "message": "Method 'testdoWork' does not have 'test_' prefix (fix available)",
    "range": {
      "end": {
        "character": 17,
        "line": 6
      },
      "start": {
        "character": 7,
        "line": 6
      }
    },
    "relatedInformation": [],
    "severity": 2,
    "source": "clang-tidy"
   }
 ],
 "uri": "file:///home/ivanmurashko/clangbook/helper.hpp",
 "version": 11
}
\end{shell}

\begin{center}
图8.34:Clangd向VS Code发送(publishDiagnostics通知)
\end{center}

如图所示(第11行),问题的修复也是可用的。在IDE中应用Clang-Tidy修复是一个惊人的机会。让我们探索一下这个功能是如何通过LSP实现的。

\mySamllsection{在IDE中应用修复}

修复可以在IDE中应用,该功能是通过"textDocument/codeAction"方法提供的。VS Code使用这个方法来提示Clangd计算特定文档和范围的命令。命令的最重要部分在以下示例中提供:

\begin{shell}
{
  "id": 98,
  "jsonrpc": "2.0",
  "method": "textDocument/codeAction",
  "params": {
    "context": {
      "diagnostics": [
      {
        "code": "misc-methodrename",
        ...
        "range": ...,
        ...
      },
    ...
  }
}
\end{shell}

\begin{center}
图8.35:VS Code向Clangd发送(textDocument/codeAction请求)
\end{center}

请求的最重要部分在第7-11行,我们可以看到原始诊断通知的副本。这些信息将被用来检索由激活检查中的clang::FixItHint提供的必要的文档修改。因此,Clangd可以响应描述所需修改的动作:

\begin{shell}
{
  "id": 98,
  "jsonrpc": "2.0",
  "result": [
  {
    "diagnostics": [
    ...
    ],
    "edit": {
      "changes": {
        "file:///home/ivanmurashko/clangbook/helper.hpp": [
        {
          "newText": "test_",
          "range": {
            "end": {
              "character": 7,
              "line": 6
            },
            "start": {
              "character": 7,
              "line": 6
            }
          }
        }
        ...
      }
    ]
  }
\end{shell}

\begin{center}
图8.36:Clangd向VS Code返回(codeAction响应)
\end{center}

图8.36中的"edit"字段是响应的最重要部分,因为它描述了要应用于原始文本的更改。

与Clang-Tidy的集成可以在不进行额外计算的情况下实现,因为Clangd核心为了导航和诊断目的构建了AST(抽象语法树)。AST可以作为Clang-Tidy检查的种子,从而消除了运行单独的Clang-Tidy可执行文件以从代码检查器检索消息的需要。这并不是Clangd进行的唯一优化;现在让我们来看另一个Clangd性能优化的例子。




