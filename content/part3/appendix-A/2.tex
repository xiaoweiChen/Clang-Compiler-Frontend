生成 compile\_commands.json 文件有多种方式。例如,构建系统 CMake 内置了对生成编译数据库的支持。一些工具还可以从 Makefile 或其他构建系统中生成这个文件。甚至还有像 Bear 和 intercept-build 这样的工具,可以通过截取实际运行的编译命令来生成 CDB。

尽管这个术语通常与 Clang 和基于 LLVM 的工具相关联,但这个概念本身更为通用,理论上可以被任何需要理解一组源文件编译设置的工具使用。我们将从使用 CMake 这款最受欢迎的构建系统之一来生成 CDB 开始。

\mySamllsection{使用 CMake 生成 CDB}

使用 CMake 生成 CDB 需要几个步骤:

\begin{enumerate}
\item
首先,打开一个终端或命令提示符,并导航到你的项目根目录。

\item
然后,运行带有 -DCMAKE\_EXPORT\_COMPILE\_COMMANDS=ON 选项的 CMake,这会指示 CMake 创建一个 compile\_commands.json 文件。这个文件包含了你项目中所有源文件的编译命令。

\item
在用 CMake 配置项目之后,你可以在运行配置命令的同一目录中找到 compile\_commands.json 文件。
\end{enumerate}

正如我们之前注意到的,LLVM 默认创建了 CDB。这是可行的,因为 llvm/CMakeLists.txt 包含以下设置:

\begin{cmake}
# Generate a CompilationDatabase (compile_commands.json file) for our build,
# for use by clang_complete, YouCompleteMe, etc.
set(CMAKE_EXPORT_COMPILE_COMMANDS 1)
\end{cmake}

\begin{center}
图 9.3:来自 llvm/CMakeLists.txt 的 LLVM-18.x CMake 配置
\end{center}

即,它默认设置了 CDB 生成。

\mySamllsection{使用 Ninja 生成 CDB}

Ninja 也可以用来生成 CDB。我们可以使用 Ninja 的一个子工具 “compdb” 将 CDB 输出到标准输出。要运行这个子工具,我们在 Ninja 中使用 -t <subtool> 命令行选项。因此,我们将使用以下命令通过 Ninja 生成 CDB:

\begin{shell}
$ ninja -t compdb > compile_commands.json
\end{shell}

\begin{center}
图 9.4:使用 Ninja 创建 CDB
\end{center}

这个命令指示 Ninja 生成 CDB 信息,并将其保存在 compile\_commands.json 文件中。

生成的编译数据库可以与我们在本书中描述的不同 Clang 工具一起使用。让我们看看两个最有价值的示例,包括 Clang-Tidy 和 Clangd。










