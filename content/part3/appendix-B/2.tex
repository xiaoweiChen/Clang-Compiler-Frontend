预编译头文件(PCH)是Clang的一个特性,旨在提高Clang前端性能。其基本思想是为头文件创建一个AST(抽象语法树)并在编译包含该头文件的源文件时重用这个AST。

生成预编译头文件很简单[5]。假设您有以下头文件,header.h:

\begin{cpp}
#pragma once

void foo() {
}
\end{cpp}

\begin{center}
图 10.1:要编译为PCH的头文件
\end{center}

您可以使用以下命令为其生成PCH:

\begin{shell}
$ <...>/llvm-project/install/bin/clang -cc1 -emit-pch        \
                                      -x c++-header header.h \
                                      -o header.pch
\end{shell}

在这里,我们使用-x c+±header选项来指定头文件应该被视为C++头文件。输出文件将被命名为header.pch。

仅仅生成预编译头文件是不够的;您需要开始使用它们。一个典型的包含该头的C++源文件可能如下所示:

\begin{cpp}
#include "header.h"

int main() {
  foo();
  return 0;
}
\end{cpp}

\begin{center}
图 10.2:包含header.h的源文件
\end{center}

如您所见,头文件是以下方式包含的:

\begin{cpp}
#include "header.h"
\end{cpp}

\begin{center}
图 10.3:头文件header.h的包含
\end{center}

默认情况下,Clang不会使用PCH,并且您需要通过以下命令明确指定它:

\begin{shell}
$ <...>/llvm-project/install/bin/clang -cc1 -emit-obj        \
                                     -include-pch header.pch \
                                     main.cpp -o main.o
\end{shell}

在这里,我们使用-include-pch来指定包含的预编译头文件:header.pch。

您可以使用调试器检查这个命令,它将给出以下输出:

\begin{shell}
$ lldb <...>/llvm-project/install/bin/clang -- -cc1 -emit-obj -include-pch header.pch main.cpp -o main.o
 ...
 (lldb) b clang::ASTReader::ReadAST
 ...
 (lldb) r
 ...
 -> 4431   llvm::TimeTraceScope scope("ReadAST", FileName);
    4432
    4433   llvm::SaveAndRestore SetCurImportLocRAII(CurrentImportLoc, ImportLoc);
    4434   llvm::SaveAndRestore<std::optional<ModuleKind>> SetCurModuleKindRAII(
 (lldb) p FileName
 (llvm::StringRef)  (Data = "header.pch", Length = 10)
\end{shell}

\begin{center}
图 10.4:在clang::ASTReader::ReadAST处加载预编译头文件
\end{center}

从这个例子中,您可以看到Clang从预编译头文件中读取AST。重要的是要注意,预编译头文件是在解析之前读取的,这使得Clang能够在解析主源文件之前从头文件中获得所有符号。这使得显式头文件包含变得不必要。因此,您可以删除源文件中的\#include "header.h"指令并成功编译。

如果没有预编译头文件,这是不可能的,您会遇到以下编译错误:

\begin{shell}
main.cpp:4:3: error: use of undeclared identifier 'foo'
   4 |   foo();
     |   ^
1  error generated.
\end{shell}

\begin{center}
图 10.5:由于缺少包含而产生的编译错误
\end{center}

值得注意的是,只有一个–include-pch选项会被处理;所有其他选项都会被忽略。这反映了一个事实,即一个翻译单元只能有一个预编译头文件。另一方面,一个预编译头文件可以包含另一个预编译头文件。这种功能称为链式预编译头文件\footnote{LLVM Community. Precompiled Header and Modules Internals. URL \url{https://clang.llvm.org/docs/PCHInternals.html}},因为它创建了一个依赖链,其中一个预编译头文件依赖于另一个预编译头文件。

预编译头文件的使用不仅限于常规编译。正如我们在图8.38中看到的,Clangd中的AST构建,预编译头文件被积极用于性能优化,作为包含头文件的预编译部分的高速缓存占位符。

预编译头文件是一种长期使用的技术,但它们有一些局限性。最重要的限制之一是只能有一个预编译头文件,这在实际项目中大大限制了PCH的使用。模块解决了一些与预编译头文件相关的问题。让我们来探讨它们。
